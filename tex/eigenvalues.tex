\chapter{Finding Eigenvalues and Vectors}

\begin{problem}
  \textsc{Given:}\enspace\ignorespaces%
an $n\times n$ matrix $A$ over $\RR$.

\textsc{Find:}\enspace\ignorespaces%
an eigenvalue $\lambda$ and possibly its eigenvector $\vec{v}$.
\end{problem}

Recall that the characteristic polynomial of an $n\times n$ matrix $A$
over a field $\FF$ is the polynomial defined by
\begin{equation}
p(\lambda):=\det(A-\lambda I).
\end{equation}
Finding eigenvalues amounts to finding the roots of an $n\times n$
polynomial, which we studied in Chapter~\ref{ch:root-finding-single-var}.
We want to find all $\lambda$ such that $p(\lambda)=0$. However, the
naive way to calculate the determinant
\begin{equation}
\det(A) = \sum_{\sigma\in S_{n}}\left(\sgn(\sigma)\prod^{n}_{i=1}a_{i,\sigma(i)}\right)
\end{equation}
would require $n$ multiplications in $n!$ terms, so there would be $n!$
addition/subtraction operations and $n!n$ multiplication
operations. This\dots this is horrible.

\begin{fact}
Similar matrices have identical eigenvalues: if $A=XBX^{-1}$, then
$\spec(A)=\spec(B)$. 
\end{fact}

\begin{theorem}[Gershgorin circle]
Let $A$ be an $n\times n$ real matrix, let $R_{i}$ be the circle in
$\CC$ centered at $a_{ii}$ with radius $\sum_{j\neq i}|a_{ij}|$, that is
to say,
\begin{equation*}
R_{i} = \{z\in\CC\mid |z-a_{ii}|<\sum^{n}_{j\neq i}|a_{ij}|\}.
\end{equation*}
Then the eigenvalues of $A$ are contained in $R=\bigcup_{i}R_{i}$, and
the union of any of these circles that do not intersect the remaining
$n-k$ must contain precisely $k$ (counting multiplicity) of the eigenvalues.
\end{theorem}

This gives us a way to give some estimate for the eigenvalues of a
matrix.

\begin{example}
Consider the matrix
\begin{equation}
A = \begin{pmatrix}4 & 1 & 1\\
 0 & 2 & -1\\
-2 & 0 &  9
\end{pmatrix}.
\end{equation}
We find its Gershgorin discs are:
\begin{enumerate}
\item $R_{1} = \{z\in\CC\mid |z-4|\leq|1|+|1|=2\}$
\item $R_{2} = \{z\in\CC\mid |z-2|\leq|0|+|-1|=1\}$
\item $R_{3} = \{z\in\CC\mid |z-9|\leq|-2|+|0|=2\}$
\end{enumerate}
Therefore the spectral radius of $A$ satisfies the bounds $7\leq\rho(A)\leq11$.
\end{example}

\section{Power Method}

Assume $A$ is an $n\times n$ matrix, and suppose $\lambda_{1}$, \dots,
$\lambda_{n}$ are eigenvalues of $A$. Let $\vec{x}_{1}$, \dots,
$\vec{x}_{n}$ be the corresponding eigenvectors for $A$, and assume they
form a basis for $\CC^{n}$. Now assume that
\begin{equation*}
\abs{\lambda_{1}}>\abs{\lambda_{2}}\geq\abs{\lambda_{3}}\cdots\geq\abs{\lambda_{n}}\geq0
\end{equation*}
The power method will give us the \emph{dominant} eigenvalue
$\lambda_{1}$ and its associated eigenvector $\vec{x}_{1}$. How?

Well, the basic idea is to start with some initial guess
$\vec{z}^{(0)}$ for the eigenvector $\vec{x}_{1}$, then perform the
iteration
\begin{equation}
\vec{w}^{(m+1)}:=A\vec{z}^{(m)},\quad\mbox{and}\quad
\vec{z}^{(m+1)}:=\frac{\vec{w}^{(m+1)}}{\|\vec{w}^{(m+1)}\|}.
\end{equation}
The claim is that $\lim_{m\to\infty}\vec{z}^{(m+1)}=\vec{x}_{1}$. How
can we see this? Well, we can consider the first few iterates for
$\vec{z}^{(m)}$ and $\vec{w}^{(m)}$:
\begin{subequations}
\begin{align}
\vec{z}^{(0)} &= \mbox{guess}\\
\vec{w}^{(1)} &= A\vec{z}^{(0)}\\
\vec{z}^{(1)} &= \frac{\vec{w}^{(1)}}{\|\vec{w}^{(1)}\|} = \frac{A\vec{z}^{(0)}}{\|A\vec{z}^{(0)}\|}\\
\vec{w}^{(2)} &= A\vec{z}^{(1)}\\
\vec{z}^{(2)} &= \frac{\vec{w}^{(2)}}{\|\vec{w}^{(2)}\|}
= \left.\frac{A^{2}\vec{z}^{(0)}}{\|A\vec{z}^{(0)}\|} \middle/
\left\|\frac{A^{2}\vec{z}^{(0)}}{\|A\vec{z}^{(0)}\|}\right\|\right.
= \frac{A^{2}\vec{z}^{(0)}}{\|A^{2}\vec{z}^{(0)}\|}
\end{align}
We see that the $m^{\text{th}}$ iterate would give us
\begin{equation}
\vec{z}^{(m)} = \frac{A^{m}\vec{z}^{(0)}}{\|A^{m}\vec{z}^{(0)}\|}.
\end{equation}
\end{subequations}
Now why does this work? Let's neglect the normalization for the moment,
since it won't affect answering why the method works.
Expand $\vec{z}^{(0)}$ as a linear combination
of the eigenbasis $\vec{x}_{1}$, \dots, $\vec{x}_{n}$:
\begin{subequations}
\begin{align}
\vec{z}^{(0)} &= \alpha_{1}\vec{x}_{1}+\cdots+\alpha_{n}\vec{x}_{n}\\
\vec{z}^{(1)} &= A\vec{z}^{(0)}= \alpha_{1}\lambda_{1}\vec{x}_{1}+\cdots+\alpha_{n}\lambda_{n}\vec{x}_{n}\\
\vec{z}^{(2)} &= A^{2}\vec{z}^{(0)}= \alpha_{1}\lambda_{1}^{2}\vec{x}_{1}+\cdots+\alpha_{n}\lambda_{n}^{2}\vec{x}_{n}\\
\vec{z}^{(m)} &= A^{m}\vec{z}^{(0)}= \alpha_{1}\lambda_{1}^{m}\vec{x}_{1}+\cdots+\alpha_{n}\lambda_{n}^{m}\vec{x}_{n}
\end{align}
then for $\vec{z}^{(m)}$, we can factorize out $\lambda_{1}^{m}$
\begin{equation}
\vec{z}^{(m)} = \lambda_{1}^{m}\left(\alpha_{1}\vec{x}_{1}+\alpha_{2}\frac{\lambda_{2}^{m}}{\lambda_{1}^{m}}\vec{x}_{2}+\cdots+\alpha_{n}\frac{\lambda_{n}^{m}}{\lambda_{1}^{m}}\vec{x}_{n}\right),
\end{equation}
and since $\lambda_{1}$ is dominant, this is approximately equal to
\begin{equation}
\vec{z}^{(m)} \approx \lambda_{1}^{m}\alpha_{1}\vec{x}_{1}.
\end{equation}
Then we can obtain $\lambda_{1}$ and $\vec{x}_{1}$, as desired.
\end{subequations}

More explicitly, let $\lambda_{1}^{(k)}$ be the $k^{\text{th}}$
approximation of $\lambda_{1}$, let $\vec{x}^{(k)}$ be the associated
eigenvector. We begin with $\vec{x}^{(0)}$ such that
$\|\vec{x}^{(0)}\|_{\infty}=1$ (say, the vector consisting of all
components equal to 1). We have the temporary variable,
\begin{equation}
\vec{y} = A\vec{x}^{(k)}.
\end{equation}
We then find the smallest index $p$ such that
\begin{equation}
y_{p} = \|\vec{y}\|_{\infty}.
\end{equation}
Then we set
\begin{equation}
\lambda_{1}^{(k+1)}=y_{p},
\end{equation}
and
\begin{equation}
\vec{x}^{(k+1)} = \frac{\vec{y}}{y_{p}}.
\end{equation}
This is the iterative method.

\begin{algorithm}\label{alg:eigenvalues:power-method}
  \caption{Power method for matrix $A$}
  \begin{algorithmic}[1]
    \Require $A=(a_{ij})$ is an $n\times n$ matrix with dominant eigenvalue
    \Require $\vec{x}_{\text{init}}$ is some initial guess $n$ vector
    \Require $1>\varepsilon_{\text{tol}}>0$ is the error tolerance
    \Require $N_{\text{max}}>0$ is the maximum number of iterations
    \Ensure $\vec{x}$ is an eigenvector
    \Ensure $\lambda$ is the dominant eigenvalue of $A$
    \Function{PowerMethod}{$A$, $\vec{x}_{\text{init}}$, $\varepsilon_{\text{tol}}$, $N_{\text{max}}$}
    \State $\vec{x}\gets\vec{x}_{\text{init}}$ \algorithmiccomment{initialize}
    \State Find the smallest $p\in\NN$ such that $p\leq n$ and $|x_{p}|=\|\vec{x}\|_{\infty}$
    \State $\vec{x}\gets \vec{x}_{\text{init}}/x_{p}$
    \For{$k=1$, \dots, $N_{\text{max}}$}
      \State $\vec{y}\gets A\vec{x}$
      \State Find the smallest $p\in\NN$ such that $p\leq n$ and $|y_{p}|=\|\vec{y}\|_{\infty}$
      \State $\lambda\gets y_{p}$
      \If{$\lambda = 0$}
        \State\Fail ``Matrix $A$ has a zero eigenvalue, guess a different $x_{\text{init}}$''
      \EndIf
      \State $r\gets \|\vec{x}-\vec{y}/\lambda\|_{\infty}$
      \State $\vec{x}\gets\vec{y}/\lambda$
      \If{$|r|<\varepsilon_{\text{tol}}$}
        \State\Return $(\lambda,\vec{x})$
      \EndIf
    \EndFor
    \State\Fail ``Maximum number of iterations exceeded''
  \EndFunction
\end{algorithmic}
\end{algorithm}
