\chapter{Polynomial Interpolation}\label{ch:polynomial-interpolation}

\begin{problem}
  \textsc{Given:} a bunch of observations as ordered pairs
  $(x_{1},y_{1})$, \dots, $(x_{n},y_{n})$ where $x_{i}\neq x_{j}$ for
  $i\neq j$.

  \textsc{Produce:} a polynomial $P(x)$ such that it passes through the
  observed points $P(x_{j})=y_{j}$ for each $j=1,\dots,n$.
\end{problem}

Why produce a polynomial? By Weierstrass' approximation theorem, it
suffices to work with polynomial approximations.

\section{Lagrange Interpolation}

\begin{ex}
  Suppose we have two observations $(x_{0},y_{0})$ and $(x_{1},y_{1})$.
  We can construct a function
  \begin{equation}
    L_{0}(x) = \frac{x-x_{1}}{x_{0}-x_{1}}
  \end{equation}
  such that $L_{0}(x_{0})=1$ and $L_{0}(x_{1})=0$. Similarly, we can
  construct 
  \begin{equation}
    L_{1}(x) = \frac{x-x_{0}}{x_{1}-x_{0}}
  \end{equation}
  which satisfies $L_{1}(x_{1})=1$ whereas $L_{1}(x_{0})=0$. Thus
  \begin{equation}
    P_{2}(x) = L_{0}(x)y_{0} + L_{1}(x)y_{1}
  \end{equation}
  is a polynomial which passes through the observations.
\end{ex}

\begin{ex}
  Consider now the case when we have three observations
  $(x_{0},y_{0})$, $(x_{1},y_{1})$,  $(x_{2},y_{2})$. We can construct
  similar functions $L_{i}(x_{j})=0$ if $i\neq j$ and $L_{i}(x_{i})=1$.
  We'd have
  \begin{subequations}
  \begin{equation}
    L_{0}(x) = \frac{x - x_{1}}{x_{0}-x_{1}}\frac{x - x_{2}}{x_{0} - x_{2}}
  \end{equation}
  for $L_{0}(x_{0})=1$ and vanishes on the other observations,
  \begin{equation}
    L_{1}(x) = \frac{x - x_{0}}{x_{1}-x_{0}}\frac{x - x_{2}}{x_{1} - x_{2}}
  \end{equation}
  and
  \begin{equation}
    L_{2}(x) = \frac{x - x_{0}}{x_{2}-x_{0}}\frac{x - x_{1}}{x_{2} - x_{1}}.
  \end{equation}
  \end{subequations}
  Then we have
  \begin{equation}
    P(x) = y_{0}L_{0}(x) + y_{1}L_{1}(x) + y_{2}L_{2}(x)
  \end{equation}
  as the approximation.
\end{ex}

We can generalize further to $n$ observations, giving us the following interpolation:

\begin{defn}\label{defn:interpolation:lagrange:lagrange-interpolating-polynomial}
  Let $(x_{1}, y_{1})$, \dots, $(x_{n}, y_{n})$ be $n > 0$ observations.
  Then the \define{Lagrange Interpolating Polynomial} is the degree $n$
  polynomial
  \begin{subequations}
  \begin{equation}
    P(x) = \sum^{n}_{i=1}y_{i}L_{i,n}(x)
  \end{equation}
  where the \define{Lagrange Basis Polynomials} are
  \begin{equation}
    L_{i,n}(x) = \prod^{n}_{\substack{j=1\\ j\neq i}}\frac{x - x_{j}}{x_{i}-x_{j}}.
  \end{equation}
  \end{subequations}
\end{defn}

\begin{ex}\label{ex:interpolation:lagrange:random-example}
  Find the Lagrange interpolation passing through the
  points\footnote{These were digits based on the time of writing.} $(1,0)$,
  $(9,12)$, $(3,6)$.

  We find
  \begin{subequations}
    \begin{align}
      P(x)
      &= 0 L_{1,3}(x) + 12 L_{2,3}(x) + 6 L_{3,3}(x)\\
      &= 12\frac{(x - 1)(x - 3)}{(9 - 1)(9 - 3)}
       + 6 \frac{(x - 1)(x - 9)}{(3 - 1)(3 - 9)}\\
      &= \frac{x^{2} - 4x + 3}{4} - \frac{x^{2} - 10x + 9}{2}.
    \end{align}
  \end{subequations}
  Hence, when expanded out,
  \begin{equation}
    P(x) = \frac{-1}{4}x^{2} + 4x - \frac{15}{4}
  \end{equation}
  is the Lagrange interpolation polynomial.
\end{ex}

\begin{ex}
  Let $f(x)$ be a function defined at $x_{1}=a$ and $x_{2}=b$. Find the
  Lagrange interpolation of the observations $(a, f(a))$ and $(b, f(b))$.

  We find
  \begin{equation}
    P(x) = \left(\frac{x-b}{a-b}\right)f(a) + \left(\frac{x-a}{b-a}\right)f(b)
  \end{equation}
  since
  \begin{subequations}
  \begin{equation}
    P(a) = \left(\frac{a-b}{a-b}\right)f(a) + \left(\frac{0}{b-a}\right)f(b)=f(a)
  \end{equation}
  and
  \begin{equation}
    P(a) = \left(\frac{0}{a-b}\right)f(a) + \left(\frac{b-a}{b-a}\right)f(b)=f(b).
  \end{equation}
  \end{subequations}
  Observe, when we expand out the interpolating polynomial, we have
  \begin{equation}
    P(x) = \frac{f(b) - f(a)}{b - a}x +\frac{bf(a) - af(b)}{b - a}.
  \end{equation}
  The slope matches intuition, the $y$-intercept may be surprising.
\end{ex}

\begin{ex}\label{ex:interpolation:lagrange:sine}
  Suppose we want to approximate $\sin(x)$ by a polynomial. We can do
  this using Lagrange polynomial interpolation operating on the
  following data: $(0,0)$, $(\pi/6,1/2)$, $(\pi/4,\sqrt{2}/2)$, $(\pi/3,\sqrt{3}/2)$, $(\pi/2,1)$.

  We find
  \begin{equation}
    P(x) = P_{0}(x) + P_{1}(x) + \dots + P_{4}(x)
  \end{equation}
  where $P_{i}(x_{j})=0$ if $i\neq j$ and $P_{i}(x_{i})=y_{i}$. Then we
  find
  \begin{subequations}
  \begin{equation}
    P_{0}(x) = 0
  \end{equation}
  \begin{equation}
    P_{1}(x) = \frac{648}{\pi^{4}}x(x - \pi/2)(x - \pi/3)(x - \pi/4)
  \end{equation}
  \begin{equation}
    P_{2}(x) = \frac{1152\sqrt{2}}{\pi^{4}}x(x - \pi/2)(x - \pi/3)(x - \pi/6)
  \end{equation}
  \begin{equation}
    P_{3}(x) = \frac{648\sqrt{3}}{\pi^{4}}x(x - \pi/2)(x - \pi/4)(x - \pi/6)
  \end{equation}
  \begin{equation}
    P_{4}(x) = \frac{144}{\pi^{4}}x(x - \pi/3)(x - \pi/4)(x - \pi/6).
  \end{equation}
  \end{subequations}
  We can add these together, expand, then collect coefficients of powers
  of $x$:
  \begin{equation}\label{eq:interpolation:polynomial:sine}
    \begin{split}
      P(x)
      &= (50 - 64\sqrt{2} + 27\sqrt{3})\frac{x}{2\pi}
      - (217 - 352\sqrt{2} + 162\sqrt{3})\frac{x^{2}}{\pi^{2}}\\
      &\quad+ (33 - 64\sqrt{2} + 33\sqrt{3})\frac{18}{\pi^{3}}x^{3}
      - (7 - 16\sqrt{2} + 9\sqrt{3})\frac{72}{\pi^{4}}x^{4}
    \end{split}
  \end{equation}
  Or as single-precision arithmetic would describe
  \begin{equation}
    P(x)\approx 0.028797112 x^4-0.20434070 x^3+0.021373008 x^2+0.99562618 x.
  \end{equation}
  This can quickly get complicated.
\end{ex}

\begin{thm}[Existence and Uniqueness]
Let $(x_{0},y_{0})$, \dots, $(x_{n}, y_{n})$ be $n>1$ observations with
$x_{i}\neq x_{j}$ for $i\neq j$.

Then there exists a unique polynomial $P(x)$ of degree at most $n$ such
that $P(x_{i})=y_{i}$ for $i=0,\dots,n$.
\end{thm}

\begin{proof}
  (1) Existence: the Lagrange interpolating polynomial from Definition~\ref{defn:interpolation:lagrange:lagrange-interpolating-polynomial}
  \begin{equation}
    P(x) = \sum^{n}_{j=0}y_{j}L_{j,n}(x)
  \end{equation}
  Then
  \begin{equation}
    L_{j,n}(x_{i}) = \delta_{i,j} = \begin{cases}1 &\mbox{if $i=j$}\\
      0 & \mbox{otherwise}
    \end{cases}
  \end{equation}
  implies
  \begin{equation}
    P(x_{i})=y_{i}
  \end{equation}
  for each $i=0,\dots,n$.

  (2) Uniqueness: Suppose $Q$ is a second polynomial of degree at most
  $n$ passing through the data. Set
  \begin{equation}
    D(x) = P(x) - Q(x)
  \end{equation}
  Thus $D(x)$ is a polynomial of degree at most $n$. Observe
  \begin{subequations}
    \begin{align}
      D(x_{i}) &= P(x_{i}) - Q(x_{i})\\
      &= y_{i} - y_{i}\\
      &= 0
    \end{align}
  \end{subequations}
  Hence $x_{i}$ is a root of $D(x)$ for each $i=0,\dots,n$.
  Thus either $\deg(D)\geq n+1$ or $D(x)=0$ is the zero polynomial.
  But we stipulated $\deg(D)\leq n$.
  Hence $D(x)=0$ is the zero polynomial, i.e., $P(x)=Q(x)$.
\end{proof}

\begin{thm}[Error]
Let $x_{0}<x_{1}<\dots<x_{n}$ be distinct and contained in $[a,b]$, let
$f\in C^{n+1}[a,b]$.

Then for each $x\in[a,b]$, a number $\xi(x)$ between $x_{0}$, $x_{1}$,
\dots, $x_{n}$ (and hence $a < \xi(x) < b$) exists with
\begin{equation}
  f(x) = P(x) + \frac{f^{(n+1)}(\xi(x))}{(n+1)!}(x-x_{0})(x-x_{1})(\dots)(x-x_{n})
\end{equation}
where $P(x)$ is the Lagrange interpolating polynomial.
\end{thm}

\begin{proof}[Proof sketch]
We write $f(x) = P(x) + R(x)$ and seek the remainder term $R(x)$. We see
that it is zero at the observations, which means it looks like
\begin{equation}
  R(x) = \rho(x)(x-x_{0})(x-x_{1})(\dots)(x-x_{n})=\rho(x)\prod^{n}_{j=0}(x-x_{j}).
\end{equation}
We then either give a sloppy inductive argument on $k=0,\dots,n$ derivatives
and invoke Rolle's theorem, or appeal to generalized Rolle's theorem, to
find $\rho(x) = f^{(n+1)}(\xi(x))/(n+1)!$.
\end{proof}

\begin{rmk}
  The proof for the error, or \emph{remainder term}, of the Lagrange
  polynomial interpolation is rather boring. There are other forms of
  the remainder, which the interested reader may find
  elsewhere\footnote{E.g., Abramowitz and Stegun, \emph{Handbook of Mathematical Functions with Formulas, Graphs, and Mathematical Tables},
  chapter 25, Eq 25.2.3, \url{https://personal.math.ubc.ca/~cbm/aands/page_878.htm}}.
  We give another expression for a bound on the remainder $R(x)$ without proof
  \begin{equation}
    |R(x)|\leq\frac{(x_{n}-x_{0})^{n+1}}{(n+1)!}\max_{x_{0}\leq x\leq x_{n}}|f^{(n+1)}(x)|.
  \end{equation}
\end{rmk}

\begin{xca}[Very hard]
  Prove, if $f(x)=1$ is the constant function, then any Lagrange
  interpolation $P(x)$ of $f(x)$ will be the constant polynomial
  $P(x)=1$.
  
  [The only proofs I could devise requires graduate level abstract
    algebra, but there may be some clever trick to prove this; for
    example, it may be easier to prove any polynomial of degree less
    than $n$ is interpolated exactly by $P(x)$ and $n$ observations.]
\end{xca}

\begin{problem}
  There are a couple problems with using Lagrange polynomial
  interpolation:
  \begin{enumerate}
  \item Adding a new observation forces us to recompute everything:
    nothing can be reused.
  \item Evaluating the Lagrange interpolation cannot easily use Horner's
    method, making it even more inefficient.
  \end{enumerate}
\end{problem}

\begin{rmk}
  For decades, Lagrange interpolation was seen as an intuitive
  introduction to polynomial interpolation, only for pedagogical
  purposes to give us divided difference methods.
  In Acton's \emph{Numerical Methods That [Usually] Work}, we read,
  ``Lagrangian interpolation is praised for analytic utility and beauty
  but deplored for numerical practice.''
  Only in 2004 did Berrut and Trefethen~\cite{berrut2004barycentric}
  rescue Lagrange interpolation with a brilliant strategy, which we will
  cover later.
\end{rmk}

% TODO: prove Newton's divided differences works
\section{Divided Differences}

The basic idea is we're given $n+1$ observations $(x_{0},y_{0})$,
$(x_{1},y_{1})$, \dots, $(x_{n}, y_{n})$ with $x_{i}\neq x_{j}$ for
$i\neq j$. We suppose $x_{0} < x_{1} < \dots < x_{n}$. What to do? We
can construct a table:
\begin{center}
  \begin{tabular}{c|c}
    $x_{0}$ & $f[x_{0}]=y_{0}$\\
    $x_{1}$ & $f[x_{1}]=y_{1}$\\
    \vdots & \vdots\\
    $x_{n}$ & $f[x_{n}]=y_{n}$
  \end{tabular}
\end{center}
The strategy is to construct a polynomial
\begin{equation}
  P(x) = c_{0} + c_{1}(x - x_{0}) + c_{2}(x - x_{0})(x - x_{1}) + \dots
  + c_{n}(x - x_{0})(x - x_{1})(\dots)(x - x_{n})
\end{equation}
which can be applied to Horner's method. The only problem: determine the
unknown coefficients $c_{0}$, $c_{1}$, \dots, $c_{n}$.

We begin by observing
\begin{equation}
  P(x_{0}) = c_{0} = y_{0}.
\end{equation}
Then for the linear term
\begin{equation}
  P(x_{1}) = y_{0} + c_{1}(x_{1}-x_{0}) = y_{1}.
\end{equation}
We can rearrange terms to find
\begin{equation}
  c_{1} = \frac{y_{1} - y_{0}}{x_{1} - x_{0}}.
\end{equation}
Now examining the quadratic term:
\begin{equation}
  P(x_{2}) = y_{0} + \frac{y_{1} - y_{0}}{x_{1} - x_{0}}(x_{2} - x_{0})
  + c_{2}(x_{2} - x_{0})(x_{2} - x_{1}) = y_{2}.
\end{equation}
We can subtract $y_{0}$ from both sides, and some algebra gives
\begin{subequations}
\begin{equation}
  \frac{y_{1} - y_{0}}{x_{1} - x_{0}}(x_{2} - x_{1} + x_{1} - x_{0})
  + c_{2}(x_{2} - x_{1})(x_{2} - x_{0}) = y_{2} - y_{0},
\end{equation}
which simplifies to
\begin{equation}
  \frac{y_{1} - y_{0}}{x_{1} - x_{0}}(x_{2} - x_{1}) + (y_{1} - y_{0})
  + c_{2}(x_{2} - x_{1})(x_{2} - x_{0}) = y_{2} - y_{0}.
\end{equation}
Subtracting $(y_{1}-y_{0})$ from both sides, then factoring the
left-hand side
\begin{equation}
  \left(\frac{y_{1} - y_{0}}{x_{1} - x_{0}}
  + c_{2}(x_{2} - x_{0})\right)(x_{2} - x_{1}) = y_{2} - y_{1},
\end{equation}
Dividing both sides by $x_{2}-x_{1}$, then subtraction gives
\begin{equation}
  c_{2}(x_{2} - x_{0}) = \frac{y_{2}-y_{1}}{x_{2}-x_{1}} - \frac{y_{1} - y_{0}}{x_{1} - x_{0}}.
\end{equation}
\end{subequations}
Dividing through by $x_{2}-x_{0}$ gives us
\begin{equation}
  c_{2} = \frac{\displaystyle\frac{y_{2}-y_{1}}{x_{2}-x_{1}} - \frac{y_{1} - y_{0}}{x_{1} - x_{0}}}{x_{2} - x_{0}}.
\end{equation}
This motivates the following notions of divided differences:
\begin{enumerate}
\item The zeroth differences $f[x_{j}] = y_{j}$
\item The first differences $f[x_{j},x_{j+1}] = (f[x_{j+1}]-f[x_{j}])/(x_{j+1}-x_{j})$
\item The second differences $f[x_{j},x_{j+1},x_{j+2}] = (f[x_{j+1},x_{j+2}]-f[x_{j},x_{j+1}])/(x_{j+2}-x_{j})$
\item The $k^{\text{th}}$ differences
  $f[x_{j},x_{j+1},x_{j+2},\dots,x_{j+k}] = \displaystyle\frac{(f[x_{j+1},x_{j+2},\dots,x_{j+k}]-f[x_{j},x_{j+1},\dots,x_{j+k-1}])}{x_{j+k}-x_{j}}$
\end{enumerate}
Then Newton's divided difference method gives us the interpolating polynomial:
\begin{equation}
  \begin{split}
  P(x) &= f[x_{0}] + f[x_{0},x_{1}](x - x_{0}) + \dots +
  f[x_{0},\dots,x_{k}](x-x_{0})(\dots)(x-x_{k-1})+\dots\\
  &\qquad+f[x_{0},\dots,x_{n}](x-x_{0})(\dots)(x-x_{n-1}).
  \end{split}
\end{equation}
Really, this is for \emph{forward differences}. The generic divided
differences method constructs the table:
\begin{center}
  \begin{tabular}{c|ccccc}
    $x_{0}$ & $f[x_{0}]=y_{0}$& & & & \\
           &                 & $f[x_{0},x_{1}] = \frac{f[x_{1}]-f[x_{0}]}{x_{1}-x_{0}}$& & & \\
    $x_{1}$ & $f[x_{1}]=y_{1}$ &    & $f[x_{0},x_{1},x_{2}]$ &  & \\
           &    & $f[x_{1},x_{2}]$& & $\ddots$ & \\
    \vdots & \vdots & \vdots & \vdots & $\dots$ &  $f[x_{0},x_{1},\dots,x_{n}]$ \\
           &    & $f[x_{n-1},x_{n-2}]$& & \reflectbox{$\ddots$} & \\
    $x_{n-1}$ & $f[x_{n-1}]=y_{n-1}$ &   & $f[x_{n-2},x_{n-1},x_{n}]$ &  &\\
           &                 & $f[x_{n-1},x_{n}] = \frac{f[x_{n}]-f[x_{n-1}]}{x_{n}-x_{n-1}}$& & & \\
    $x_{n}$ & $f[x_{n}]=y_{n}$ && 
  \end{tabular}
\end{center}
Then we take a ``path'' in this table, using the coefficients and
factors accordingly. If we move ``up'' in our path, it costs us a sign.
This is rather abstract, let's look at a couple of examples.

First, we can recover Newton's divided differences from this table by
moving along the top of the table.

We could have moved along the ``bottom path'' instead, taking
\begin{equation}
  \begin{split}
  \widetilde{P}(x) &= f[x_{n}] - f[x_{n-1},x_{n}](x - x_{n})
  - f[x_{n-2},x_{n-1},x_{n}](x - x_{n-1})(x - x_{n}) - \dots \\
 &\qquad - f[x_{0}, x_{1}, \dots, x_{n}](x - x_{0})(x - x_{1})(\dots)(x - x_{n}).
  \end{split}
\end{equation}


\begin{algorithm}\label{alg:divided-diff:newton}
  \caption{Newton's Divided-Differences Method for polynomial interpolation}
  \begin{algorithmic}[1]
    \Require $(x_{0},y_{0})$, \dots, $(x_{n},y_{n})$ are $n+1$ distinct observations
    \Ensure the result are numbers $F_{0,0}$, $F_{1,1}$, \dots,
    $F_{n,n}$ where $P(x) = \sum_{j=0}^{n} F_{j,j}\prod^{j-1}_{k=0}(x-x_{k})$
    \Function{NewtonInterpolation}{$x$, $y$}
    \State Initialize $F_{0,\cdot} \gets y_{\cdot}$
    \For{$i=1,n$}
      \For{$j=1,i$}
        \State Set $F_{i,j}\gets\frac{F_{i,j-1} - F_{i-1,j-1}}{x_{i}-x_{i-j}}$
      \EndFor
    \EndFor
    \State\Return $(F_{0,0}, F_{1,1}, \dots, F_{n,n})$
    \EndFunction
  \end{algorithmic}
\end{algorithm}

\begin{ex}
  Find the Newton divided-differences polynomial using the data from Example~\ref{ex:interpolation:lagrange:random-example}: $(1,0)$,
  $(9,12)$, $(3,6)$.

  We can setup the table:
  \begin{center}
    \begin{tabular}{c|ccc}
      $x_{0}=1$ & $f[x_{0}]=0$  &                 & \\
                &              & $f[x_{0},x_{1}]=6/2=3$ & \\
      $x_{1}=3$  & $f[x_{1}]=6$ &      & $f[x_{0},x_{1},x_{2}]=(1-3)/(9-1)=-1/4$\\
                &              & $f[x_{1},x_{2}]=6/6=1$ & \\
      $x_{2}=9$ & $f[x_{2}]=12$  &                 & 
    \end{tabular}
  \end{center}
  Then we have
  \begin{equation}
    P(x) = 0 + 3(x - 1) - \frac{1}{4}(x - 1)(x - 3).
  \end{equation}
  Does this work? Well, Lagrange polynomial interpolation gave us:
  \begin{equation}
    Q(x) = \frac{-1}{4}x^{2} + 4x - \frac{15}{4}
  \end{equation}
  whereas Newton gives us
  \begin{equation}
    P(x) % = 3x - 3 - \frac{1}{4}(x^{2} - 4x + 3)
    = \frac{-1}{4}x^{2} + 4x - \frac{15}{4}.
  \end{equation}
  The two methods agree!
\end{ex}

\begin{ex}
Let's try to approximate $\sin(x)$ using the observations $(0,0)$,
$(\pi/6,1/2)$, $(\pi/4,\sqrt{2}/2)$, $(\pi/3,\sqrt{3}/2)$, $(\pi/2,1)$.
We can compare the result to Example~\ref{ex:interpolation:lagrange:sine},
when we used Lagrange interpolation.

We have
\begin{center}
\begin{tabular}{c|ccc}
  $x_{0}=0$ & $f[x_{0}]=0$ & &  \\
  
           & & $f[x_{0},x_{1}] = \displaystyle\frac{3}{\pi}$ & \\
  
  $x_{1}=\displaystyle\frac{\pi}{6}$ & $f[x_{1}]=\displaystyle\frac{1}{2}$ & & $f[x_{1},x_{2},x_{3}]=\displaystyle\frac{12}{\pi^{2}}(\sqrt{2}-3)$  \\
  
            & & $f[x_{1},x_{2}] =\displaystyle \frac{6}{\pi}(\sqrt{2}-1)$ & \\

  $x_{2}=\displaystyle\frac{\pi}{4}$ & $f[x_{2}]=\displaystyle\frac{\sqrt{2}}{2}$ & & $f[x_{1},x_{2},x_{3}]=\displaystyle\frac{36}{\pi^{2}}(1+\sqrt{3}-2\sqrt{2})$ \\
  
            & & $f[x_{2},x_{3}] = \displaystyle\frac{6}{\pi}(\sqrt{3} - \sqrt{2})$ & \\

  $x_{3}=\displaystyle\frac{\pi}{3}$ & $f[x_{3}]=\displaystyle\frac{\sqrt{3}}{2}$ & & $f[x_{2},x_{3},x_{4}]=\displaystyle\frac{12}{\pi^{2}}(2+2\sqrt{2}-3\sqrt{3})$ \\
  
            & & $f[x_{3},x_{4}] = \displaystyle\frac{3}{\pi}(2 - \sqrt{3})$ &\\

  $x_{4}=\displaystyle\frac{\pi}{2}$ & $f[x_{4}]=1$ & & 
\end{tabular}
\end{center}
And the higher differences:
\begin{center}
  \begin{tabular}{cc}
    $f[x_{0},x_{1},x_{2},x_{3}]=\displaystyle\frac{36}{\pi^{3}}(6+3\sqrt{3}-8\sqrt{2})$&\\
    &$f[x_{0},x_{1},x_{2},x_{3},x_{4}]=\displaystyle\frac{72}{\pi^{4}}(16\sqrt{2}-7-9\sqrt{3})$\\
    $f[x_{1},x_{2},x_{3},x_{4}]=\displaystyle\frac{36}{\pi^{3}}(8\sqrt{2}-1-3\sqrt{3})$&
  \end{tabular}
\end{center}
The Newton's divided-difference polynomial gives us
\begin{equation}
  \begin{split}
  P(x) &= \frac{3}{\pi}x + \frac{12}{\pi^{2}}(\sqrt{2}-3)x(x-\pi/6)\\
  &\qquad+ \frac{36}{\pi^{3}}(6+3\sqrt{3}-8\sqrt{2}) x(x-\pi/6)(x - \pi/4)\\
  &\qquad+ \frac{72}{\pi^{4}}(16\sqrt{2}-7-9\sqrt{3})x(x-\pi/6)(x - \pi/4)(x-\pi/2)
  \end{split}
\end{equation}
We can simplify, a bit, to get
\begin{equation}
  \begin{split}
P(x) &= [36 - 32\sqrt{2} + 9\sqrt{3} - (7 - 16\sqrt{2} + 9\sqrt{3})\pi]\frac{x}{2\pi}\\
&\quad-\frac{3}{2\pi^{2}}[84 - 96\sqrt{2} + 30\sqrt{3} - 5\pi(7 - 16\sqrt{2}+9\sqrt{3})]x^{2}\\
&\quad+ \frac{4}{\pi^{3}}[9(6 - 8\sqrt{2} + 3\sqrt{3}) - 10\pi(7 - 16\sqrt{2}
  + 9\sqrt{3})]x^{3}\\
&\quad+ \frac{90}{\pi^{3}}(7 - 16\sqrt{2} + 9\sqrt{3}) x^{4}\\
&\quad- \frac{72}{\pi^{4}}(7 - 16\sqrt{2} + 9\sqrt{3}) x^{5}
  \end{split}
\end{equation}
Or, as a single-precision approximation to this polynomial:
\begin{equation}
  P(x)\approx 1.0275073 x - 0.12295329 x^{2}  + 0.021408739 x^{3}  - 0.11308600 x^{4}
+ 0.028797112 x^{5}
\end{equation}
What does Lagrange polynomial interpolation give us? Well, we can recall Eq~\eqref{eq:interpolation:polynomial:sine}
\begin{equation*}
  \begin{split}
    P_{\text{Lagrange}}(x)
    &= (50 - 64\sqrt{2} + 27\sqrt{3})\frac{x}{2\pi}
    - (217 - 352\sqrt{2} + 162\sqrt{3})\frac{x^{2}}{\pi^{2}}\\
    &\quad+ (33 - 64\sqrt{2} + 33\sqrt{3})\frac{18}{\pi^{3}}x^{3}
    - (7 - 16\sqrt{2} + 9\sqrt{3})\frac{72}{\pi^{4}}x^{4}
  \end{split}
\end{equation*}
which is clearly a different polynomial (they're of different orders!).

So which approximation is better? Well, it depends on what you mean by
``better''. But one measure of goodness is the $L^{2}$-distance between
the polynomial and $\sin(x)$ over the interval $0\leq x\leq \pi/2$. This
is a straightforward, though tedious, calculation:
\begin{equation}
  \begin{split}
  \int^{\pi/2}_{0}&(P(x)-\sin(x))^{2}\,\D x\\
 =& \frac{120960-276480\sqrt{2}+155520 \sqrt{3}}{\pi ^4}
  +\frac{-27648+65664\sqrt{2}-37584 \sqrt{3}}{\pi^3}\\
  &+\frac{-3864+8256 \sqrt{2}-4500\sqrt{3}}{\pi^2}
  +\frac{102-304 \sqrt{2}+189 \sqrt{3}}{\pi}\\
  &+7 - 16\sqrt{2} + 9\sqrt{3}
+\left(1294238-392832 \sqrt{2}+238788 \sqrt{3}-440352 \sqrt{6}\right)
  \frac{\pi }{147840}\\
&+\left(52459-13112 \sqrt{2}+8415 \sqrt{3}-19800 \sqrt{6}\right)
\frac{\pi^2}{147840}\\
&+\left(2010-560 \sqrt{2}+315 \sqrt{3}-720 \sqrt{6}\right) \frac{\pi^3}{147840}\\
\approx&4.029670120332435\times 10^{-5}
  \end{split}
\end{equation}
And for Lagrange interpolation:
\begin{equation}
  \begin{split}
  &\int^{\pi/2}_{0}(P_{\text{Lag}}(x)-\sin(x))^{2}\,\D x\\
  =& \frac{3456}{\pi ^4} \left(7-16 \sqrt{2}+9 \sqrt{3}\right)
  +\frac{216}{\pi ^3}\left(-23+64 \sqrt{2}-39 \sqrt{3}\right)\\
  &+\frac{4}{\pi ^2} \left(-217+352\sqrt{2}-162 \sqrt{3}\right)
  +\frac{\pi}{6720} \left(6948-1648 \sqrt{2}+1107 \sqrt{3}-1296 \sqrt{6}\right) \\
  &+\frac{-3-64 \sqrt{2}+54 \sqrt{3}}{\pi}\\
  \approx&3.20243293420500498\times 10^{-8}
  \end{split}
\end{equation}
So we find Newton's divided difference is slightly worse by this metric.
\end{ex}

\section{Chebyshev Interpolation}

We've seen polynomial interpolation approximate complicated functions
like $\sin(x)$. But the approximation left much to be desired. What's
the solution? Adding more observations? Sometimes this causes more
problems than it solves.

\begin{ex}[Runge phenomena]
  Runge studied the exponential function in a clever way. He took
  \begin{equation}
    u := \E^{h}-1
  \end{equation}
  and
  \begin{equation}
    v := x/h
  \end{equation}
  then examined the [truncated] binomial expansion of
  \begin{equation}
    g_{n}(x) = (1 + u)^{v} = 1 + uv + u^{2}\frac{v(v-1)}{1\cdot 2}
    +\dots+u^{n}\frac{v(v-1)(\dots)(v-(n-1))}{1\cdot2\cdot3\cdot(\dots)\cdot n}.
  \end{equation}
  It should converge to $\exp(x)$, right?
  
  Consider the function
  \begin{equation}
    f(x) = \frac{1}{1 + x^{2}}
  \end{equation}
  on the domain $x\in[-1,1]$.
\end{ex}
\section{Barycentric Lagrange Interpolation}

In recent years, numerical analysts have revisited Lagrange polynomial
interpolation. We will briefly review the conceptual underpinnings of
it. Let
\begin{equation}
  L(x) = \prod^{n}_{k=1}(x - x_{k}).
\end{equation}
Observe
\begin{equation}
  L'(x_{j}) = \left.\frac{\D L(x)}{\D x}\right|_{x=x_{j}}
  = \prod^{n}_{k\neq j}(x_{j} - x_{k}).
\end{equation}
This permits us to rewrite the Lagrange basis polynomials Eq~\eqref{eq:interpolation:lagrange:basis-polynomial} as
\begin{equation}
  L_{i,n}(x) = \frac{L(x)}{L'(x_{i})(x - x_{i})}.
\end{equation}
Let's write
\begin{equation}
  w_{j} = \frac{1}{L'(x_{j})}
\end{equation}
for the \define{Barycentric Weights}, then
\begin{equation}
  L_{i,n}(x) = L(x)\frac{w_{i}}{x - x_{i}}.
\end{equation}
Thus we have
\begin{equation}\label{eq:interpolation:barycentric-lagrange:rewrite}
  P(x) = L(x) \sum^{n}_{k=1}\frac{w_{k}}{x - x_{k}}y_{k}.
\end{equation}
This is the first half of the argument, because we really haven't done
anything yet.

The second half is to recall
Exercise~\ref{xca:interpolation:lagrange:one},
where we asked the reader to prove the Lagrange polynomial interpolation
for $g(x)=1$ is the constant polynomial $Q(x)=1$. Now, we take
Eq~\eqref{eq:interpolation:barycentric-lagrange:rewrite}
and divide it by 1. Well, first, $Q(x)$ may be written, using
Eq~\eqref{eq:interpolation:barycentric-lagrange:rewrite}, as
\begin{equation}
  Q(x) = L(x) \sum^{n}_{k=1}\frac{w_{k}}{x - x_{k}}.
\end{equation}
Then
\begin{equation}
  P(x) = \frac{P(x)}{1} = \frac{P(x)}{Q(x)}
\end{equation}
and we have
\begin{equation}
  \boxed{P(x) = \frac{\displaystyle \sum^{n}_{k=1}\frac{w_{k}}{x - x_{k}}y_{k}}{\displaystyle\sum^{n}_{k=1}\frac{w_{k}}{x - x_{k}}}}.
\end{equation}
This has 1 subtraction and 1 division in the calculation of
$w_{j}/(x - x_{j})$, there are $n$ multiplications in the numerator, and
$2n$ additions in the numberator and denominator.
Thus evaluating this would require $n$ subtractions, $2n$ additions, $n$
multiplications, and $n+1$ divisions. Compare this to divided
differences which requires $(n-1)(n-2)/2$ division operations to compute
the coefficients, and an equal number of subtractions. Reader:
divided differences requires $\sim n^{2}$ division and subtraction
operations, whereas barycentric Lagrange interpolation requires only
$\sim n$ operations!

\begin{algorithm}\label{alg:interpolation:barycentric-lagrange}
  \caption{Barycentric Lagrange Polynomial Interpolation}
  \begin{algorithmic}[1]
    \Require $(x_{0},y_{0})$, \dots, $(x_{n},y_{n})$ are $n+1$ distinct observations
    \Require some point $x$ to evaluate the polynomial at
    \Ensure the result corresponds to $P(x)$

    \Function{BarycentricLagrangeInterpolation}{$x_{j}$, $y_{j}$,x}
    \State Initialize $num\gets 0$, $den\gets 0$
    \For{$i=0,n$}
      \If{$x_{i}=x$}
      \State\Return $y$
      \Else
      \State Set $t \gets w_{i}\oslash(x \ominus x_{i})$
      \State Set $num\gets num \oplus (t\otimes y_{i})$
      \State Set $den\gets den\oplus t$
      \EndIf
    \EndFor
    \State\Return $num\oslash den$
    \EndFunction
  \end{algorithmic}
\end{algorithm}

The implementation amounts to ``following your nose'', as shown in
Algorithm~\ref{alg:interpolation:barycentric-lagrange}. Care must be
taken if trying to evaluate the Barycentric Lagrange interpolation at
$x\approx x_{i}$ for some observed point $(x_{i},y_{i})$, as it can lead
to catastrophic loss of precision. The other passing remark, if you use
a language which can return a closure (function object, lambda), then
you can transform the algorithm into a factory method by Currying.

\begin{xca}
  How many addition, subtraction, multiplication, and division
  operations occur in Algorithm~\ref{alg:interpolation:barycentric-lagrange}?
\end{xca}

\begin{xca}[Hard, open?]
  Algorithm~\ref{alg:interpolation:barycentric-lagrange} does not use
  Horner's method. Does this lead to numerical instability?
\end{xca}


