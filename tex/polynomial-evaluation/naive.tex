\section{Naive Polynomial Evaluation}

The first attempt at a solution is to just ``do the obvious thing'':
loop over the coefficients, multiply them by $x^{k}$, then sum them all
up. This is summarized in Algorithm~\ref{alg:naive-polynomial-evaluation}.
The first question we should ask:
\emph{How many operations does this algorithm use?}

\begin{algorithm}\label{alg:naive-polynomial-evaluation}
  \caption{Naive polynomial evaluation.}
  \begin{algorithmic}[1]
    \Function{NaiveEval}{$p$, $x_{0}$} \Comment{$p$ is an array}
      \State\label{alg-step:naive-polynomial-evaluation:initial-load} $r\gets p_{0}$
      \State $n\gets\Call{PolynomialOrder}{p}$
      \State $x\gets 1$
      \For{$j=1,n$}
        \State $x\gets x\otimes x_{0}$
        \State\label{alg-step:naive-polynomial-evaluation:fma} $r\gets r\oplus p_{j}\otimes x$
      \EndFor
      \State \Return $r$  
    \EndFunction
\end{algorithmic}
\end{algorithm}

\begin{thm}
  Let $p(z) = p_{n}z^{n} + \dots + p_{1}z + p_{0}$ be a polynomial of
  degree $n$. Na\"{\i}ve polynomial evaluation requires $n$ addition
  operations, $2n$ multiplication operations, and $n+1$ memory load
  operations.
\end{thm}
\begin{proof}
Each iteration of the loop has 1 addition operation, and 2
multiplication operations. This gives us a total of $n$ addition
operations and $2n$ multiplication operations.
\end{proof}

\begin{rmk}
If the computer has a fused multiplication--addition operation, then
step~\ref{alg-step:naive-polynomial-evaluation:fma} may be done in one
operation. This reduces the number of operations to $n$ additions, and
$n$ FMA operations.
\end{rmk}


\begin{ex}[Wilkinson Polynomial]
  A stress-test we may consider is the Wilkinson polynomial of degree
  20:
  \begin{equation}
    w_{20}(x) = \prod^{20}_{k=1}(x-k)
  \end{equation}
  We compute the coefficients by hand:
  \begin{equation}
    \begin{split}
      &w_{20}(x)\\
      &= x^{20}-210 x^{19}+20615 x^{18}-1256850 x^{17}+53327946 x^{16}-1672280820x^{15}\\
    &+40171771630 x^{14}-756111184500 x^{13}+11310276995381 x^{12}\\
      & -135585182899530 x^{11}+1307535010540395 x^{10}-10142299865511450 x^9\\
      &+63030812099294896 x^8-311333643161390640 x^7+1206647803780373360 x^6\\
      &-3599979517947607200 x^5+8037811822645051776 x^4-12870931245150988800 x^3\\
      & +13803759753640704000 x^2-8752948036761600000 x+2432902008176640000
    \end{split}
  \end{equation}
  Using the na\"{\i}ve polynomial evaluation, and double-precision
  arithmetic, we have the following values:
  \begin{center}
    \begin{tabular}{c|l}
      $x$ & naive($x$)\\ \hline
      1 &   $160.00000000000000$      \\
      2 &  $-6144.0000000000000$      \\
      3 &  $-138622.00000000000$      \\
      4 &  $-524288.00000000000$      \\
      5 &   $51735.000000000000$      \\
      6 &  $-14942208.000000000$      \\
      7 &  $-23677216.000000000$      \\
      \textbf{8} & \textbf{0.0000000000000000}      \\
      9 &  $-409311232.00000000$      \\
      10 &   $293601280.00000000$      \\
      $10.000999999999999$      &  $-1239859200.0000000$      \\
      11 &  $-2502688768.0000000$      \\
      12 &  $-12884901888.000000$      \\
      13 &   $11492392960.000000$      \\
      14 &  $-12113149952.000000$      \\
      15 &   $30333206528.000000$      \\
      \textbf{16} & \textbf{0.0000000000000000}      \\
      17 &   $370440929280.00000$      \\
      18 &  $-2806761127936.0000$      \\
      19 &   $1979979923456.0000$      \\
      20 &   $3848290697216.0000$      \\
    \end{tabular}
  \end{center}
  Observe the only values which correspond with the mathematical
  function are in bold (8, and 16). This is horrible, how can we do better?
\end{ex}

\begin{ex}\label{ex:polynomial-evaluation:motivation-horner}
  Consider a cubic $f(x)=2+3x+4x^{2}+5x^{3}$. Evaluate it at, say,
  $x=7$.

  Naive evaluation would do this simply by
  \begin{equation}
    \begin{split}
      f(7) &= 2 + 3\cdot7 + 4\cdot 7^{2} + 5\cdot 7^{3}\\
      &=2 + 3\cdot7 + 4\cdot7\cdot7 + 5\cdot7\cdot7\cdot7.
    \end{split}
  \end{equation}
  Observe, we can factor this by
  \begin{equation}
    f(7) = 2 + 7\cdot(3 + 4\cdot7 + 5\cdot 7\cdot7).
  \end{equation}
  Does this change anything? Well, naive evaluation had 6
  multiplications, now we have 4 multiplications. That's faster. We can
  iterate this scheme:
  \begin{equation}
    f(7) = 2 + 7(3 + 7(4 + 5\cdot7)).
  \end{equation}
  This has 3 multiplication operations, quite a bit better.
\end{ex}
\begin{lesson}
  Using a simple polynomial like this can test the software
  implementation of Algorithm~\ref{alg:naive-polynomial-evaluation}
  for obvious bugs. Although this doesn't prove the program correct, it
  catches simple errors before they can cause problems.
\end{lesson}

\subsection{*Cost of Memory Load Operations}

We could also note, when we write $p_{j}$, we really mean, ``Load the
contents of memory cell associated with $p_{j}$ into a CPU register.''
If we wanted to track this as well, we would note there is an initial
load operation in
step~\ref{alg-step:naive-polynomial-evaluation:initial-load}, and in
each iteration (in step~\ref{alg-step:naive-polynomial-evaluation:fma})
there is a load operation. Hence we have $n+1$ load operations. These
are not necessarily negligible!

On x86 architectures, {\tt gfortran}\footnote{Using GNU Fortran (Ubuntu
9.3.0-17ubuntu1-20.04) 9.3.0; the flags supplied are ``\texttt{-c -S
  -fverbose-asm -O0 -c}''. On x86, the additional flags
``\texttt{-march=native -masm=intel}'' are given.} without optimization,
when compiled for Haswell architecture, it translates $p_{j}$ lookup to
a \texttt{MOV R,R} instruction, a \texttt{CDQE} instruction,
a \texttt{MOV R,R} instruction, and a \texttt{VMOVSD} instruction.
The \texttt{VMOVSD} costs 6
cycles\footnote{\url{https://uops.info/html-instr/VMOVSD_XMM_M64.html}},
the \texttt{MOV R,R} costs 1 cycle each, and the \texttt{CDQE} costs 1
cycle. Hence together they cost 9 cycles, comparable to 3 \texttt{FADD} operations.

When cross-compiling for \textsc{Sparc64}, looking up $p_{j}$
compiles\footnote{If we had a \texttt{get\_elt(p,j) = p[j]} function, this
would require an additional {\tt ldx} operation. The
difference would cost us not just 3 cycles, but also the overhead of a
function call. The point being: on \textsc{Sparc64}, accessing an array
seems to be comparable to a floating-point addition operation in CPU cycles.}
to one {\tt ldd} operation, an {\tt ld} operation, an {\tt ldx}
operation, an {\tt sra} operation, and a {\tt sllx} operation. Appendix
B3 for the UltraSPARC T1 supplement\footnote{\url{https://www.oracle.com/technetwork/systems/opensparc/t1-09-ust1-uasuppl-draft-hp-ext-1537737.html}}
gives the following latencies for these instructions: {\tt ld} and {\tt ldd}
take 9 cycles, {\tt ldx} takes 3 cycles, {\tt sra} takes 1 cycle, and
{\tt sllx} takes 1 cycle. This sums to 23 cycles, comparable to a
floating-point addition 26 cycle latency.
% fmuld - 29, faddd - 26, fdivd 83, 

An \textsc{Arm} Cortex-A9 (with hard-float support) is more explicit
taking $p_{j}$ as two \texttt{LDR} operations, followed by an
\texttt{LSLS}, an integer \texttt{ADD} of $p$ and $j$, then a
\texttt{VLDR} operation to load that memory cell as a double-precision
floating-point. These instructions may be tallied as follows:\footnote{\url{https://developer.arm.com/documentation/ddi0388/i/cycle-timings-and-interlock-behavior/load-and-store-instructions}}
3 cycles for \texttt{LDR}, 1 cycle for \texttt{ADD}, 1 cycle for \texttt{VLDR},
1 cycle for \texttt{LSLS}. Altogether, accessing the array element costs
9 cycles. This is identical to the total time it takes for \texttt{VADD}
to add two floating-point numbers and write it back to the register
file.

For x64, Haswell architecture, using Intel syntax, it compiles to:
\begin{lstlisting}[language={[x86]Assembler}]
    mov     eax, DWORD PTR -20[rbp]     # tmp112, j
    cdqe
    mov     rdx, QWORD PTR -72[rbp]     # tmp113, p
    vmovsd  xmm0, QWORD PTR [rdx+rax*8] # _7, *p_20(D)
\end{lstlisting}
For \textsc{Sparc64}:
\begin{lstlisting}[language={[sparc64]Assembler}]
    ld   [%fp+2043], %g1 ! j, tmp143
    sra  %g1, 0, %g1     ! tmp143, _6
    ldx  [%fp+2175], %g2 ! p, tmp144
    sllx %g1, 3, %g1     ! _6,, tmp145
    ldd  [%g2+%g1], %f10 ! *p_20(D), _7
\end{lstlisting}
For ARM:
\begin{lstlisting}[language={[ARM]Assembler}]
    ldr     r2, [r7, #12]    @ tmp140, p
    ldr     r3, [r7, #52]    @ tmp141, j
    lsls    r3, r3, #3       @ tmp142, tmp141,
    add     r3, r3, r2       @ tmp143, tmp140
    vldr.64 d6, [r3]         @ _5, *p_18(D)
\end{lstlisting}
\textsc{Risc-v} compiles it to:
\begin{lstlisting}[language={[RISC-V]Assembler}]
    lw    a5,-20(s0)  # _6, j
    ld    a4,-72(s0)  # tmp106, p
    slli  a5,a5,3     #, tmp107, _6
    add   a5,a4,a5    # tmp107, tmp108, tmp106
    fld   fa4,0(a5)   # _7, *p_20(D)
\end{lstlisting}
MIPS compiles it to:
\begin{lstlisting}[language={[mips]Assembler},
  keywordstyle={[3]\color{red!50!black}}]
    lw    $3,64($fp)   # tmp225, p
    lw    $2,8($fp)    # tmp226, j
    sll   $2,$2,3      # tmp227, tmp226,
    addu  $2,$3,$2     # tmp228, tmp225, tmp227
    ldc1  $f2,0($2)    # _5, *p_18(D)
\end{lstlisting}
For MIPS I-7200\footnote{Ch. 15 of \url{https://s3-eu-west-1.amazonaws.com/downloads-mips/I7200/I7200+product+launch/MIPS_I7200_Programmers_Guide_01_20_MD01232.pdf}},
these are 2 load operations (\texttt{lw}), 2 arithmetic operations
(\texttt{sll} and \texttt{addu}), and an ``\verb#ldc1#'' operation. Load
operations cost between 2 and 3 cycles, arithmetic operations cost
between 1 and 2 cycles. Altogether, this costs between 8 to 13 cycles,
or between 2 and 3 floating-point addition operations\footnote{See,
e.g., \url{https://lauri.võsandi.com/tub/computer-architecture/mips-pipeline.html}}.

These three examples show looking up $p_{j}$ can cost anywhere between 1
and 3 floating-point addition operations.

