\section{Error Analysis}

From
Example~\ref{ex:polynomial-evaluation:horner:horner-with-base-points-wilkinson},
we saw there was error when encoding $x_{0}=10.001$ into a floating-point
number, and an error when computing $w_{20}(x_{0})$ using floating-point
numbers (i.e., even if we had $x_{0}$ exactly representable, there would
still be error with $w_{20}(x_{0})$ due to floating-point). We see
round-off error gives us
\begin{equation}
  x_{0} + \Delta x_{\text{round}} = \frac{5630062484166541}{562949953421312}
\end{equation}
where
\begin{equation}
  \begin{split}
    \Delta x_{\text{round}} &= \frac{39}{70368744177664000}\\
    &\approx5.5422333389287814497947692871094 \times10^{-16}.
  \end{split}
\end{equation}
At the same time, we have error computing $w_{20}(x_{0})$ using some
algorithm, ${\rm alg}(x_{0})$, which gives us:
\begin{equation}
  y_{0}^{*} := {\rm alg}(x_{0}) = w_{20}(x_{0}) + \Delta y.
\end{equation}
Here $\Delta y$ is the difference between \emph{the result} $y_{0}^{*}$
and \emph{the answer} $y_{0}$. Horner's method computes Wilkinson's
polynomial at $10.001$ as
\begin{equation}
  \begin{split}
  y_{0}^{*} &:= \frac{1380644536674587}{1048576}\\
  &\approx 1316685234.7131605148315429687500\dots
  \end{split}
\end{equation}
We find $x_{0}^{*}$ to be the real number closest to $x_{0}$ such that
$w_{20}(x_{0}^{*})=y_{0}^{*}$. Empirically, we find
\begin{equation}
  x_{0}^{*} = x_{0} + \Delta x \approx 10.000999999999999445651260815157\dots
\end{equation}
giving us
\begin{equation}
  \Delta x \approx 5.5434873918484323026856673893040\times 10^{-16}.
\end{equation}
In some sense, this $x_{0}^{*}$ tells us the argument which the result
is the solution for. Let's introduce some terms for describing these
quantities.
\begin{defn}
Let $\mathrm{alg}(x)$ be an algorithm approximating $f(x)$, let
$y^{*}=\mathrm{alg}(x)$ be the (real number) result, and $y=f(x)$ be the
answer. Let $x^{*}$ be the real number closest to $x$ such that
$y^{*}=f(x^{*})$.

We define the \define[Error!Backward]{(Absolute) Backwards Error} to be the quantity
\begin{equation}
  \Delta x := x^{*} - x
\end{equation}
and the \define[Error!Forward]{(Absolute) Forward Error} to be the quantity
\begin{equation}
  \Delta y := y^{*} - y.
\end{equation}
Graphically, we can picture this as:
\begin{center}
  \includegraphics{img/img-1.mps}
\end{center}
\end{defn}
\begin{rmk}[Non-Uniqueness]
  There's no reason for $\Delta x$ to be unique. We could have many
  possible $x^{*}$ such that $f(x^{*})=y^{*}$ which are equally close to
  $x$.
\end{rmk}
\begin{rmk}[History]
  James Hardy Wilkinson\index{Wilkinson, James Hardy}, the mathematician who invented the Wilkinson
  polynomial as a stress-test for numerical analysis, also developed the
  idea of backward error analysis between 1957 and 1962. See, e.g.,
  chapter 4 of
  \emph{Algebraic Eigenvalue Problem}~\cite{wilkinson1965}, or
  \cite[(\S6~\textit{et seq}.)]{wilkinson1963}.
\end{rmk}
