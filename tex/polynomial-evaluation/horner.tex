\section{Horner Evaluation}

We can generalize Example~\ref{ex:polynomial-evaluation:motivation-horner}
by evaluating
\begin{equation}
  p(z) = p_{n}z^{n} + \dots + p_{1}z + p_{0}
\end{equation}
by factoring
\begin{equation}
  p(z) = \bigl(((p_{n}z + p_{n-1})z + \dots)z + p_{1}\bigr)z + p_{0}.
\end{equation}
The naive evaluation had $p_{k}z^{k}$ require $k+1$ multiplication
operations, so there are a total of
\begin{equation}
  \sum^{n}_{k=1}k+1 = \frac{n(n+1)}{2}
\end{equation}
multiplication operations for naive evaluation. This clever
factorization scheme requires $n$ multiplication operations.
Let's explicitly describe this in Algorithm~\ref{alg:horner-polynomial-evaluation}.

\begin{algorithm}\label{alg:horner-polynomial-evaluation}
  \caption{Horner polynomial evaluation.}
  \begin{algorithmic}[1]
    \Function{HornerEval}{$p$,$x_{0}$} \Comment{$p$ is an array}
      \State$n\gets\Call{PolynomialOrder}{p}$
      \State$r\gets p_{n}$
      \For{$j=1,n$}
        \State\label{step:horner-polynomial-evaluation:loop-body}$r\gets (r\otimes x_{0}) \oplus p_{n-j}$
      \EndFor
      \State \Return $r$
    \EndFunction
\end{algorithmic}
\end{algorithm}

\begin{thm}
  Let $p(z) = p_{n}z^{n} + \dots + p_{1}z + p_{0}$ be a polynomial of
  degree $n$. Then Horner evaluation of $p$ at $x_{0}$ requires $n$
  multiplications, $n$ additions, and $n+1$ memory load operations.
\end{thm}
\begin{proof}
Observe in the loop, each iteration has 1 multiplication operation and 1
addition operation. There are $n$ iterations to the loop, giving us a
total of $n$ addition operations and $n$ multiplication operations.
Like na\"{\i}ve polynomial evaluation, there are $n+1$ memory load
operations.
\end{proof}
\begin{rmk}
  If the computer has fused multiplication--addition operations, then
  step~\ref{step:horner-polynomial-evaluation:loop-body} can be done in
  a single floating-point operation (improving accuracy quite a bit).
\end{rmk}
\begin{chunk}
  The \FORTRAN/ implementation of this:\marginpar{\texttt{polynomial.f90}}
  \lstinputlisting[language=fortran18,firstline=32,lastline=45]{src/polynomial.f90}
\end{chunk}
\begin{ex}[Wilkinson stress-test revisited]\index{Wilkinson Polynomial}
We can apply our new algorithm to $w_{20}(x)$ to see what our results
look like.
\begin{center}
  \begin{longtable}{r|r|r}
$x$  & Naive Evaluation      & Horner Evaluation\\\hline
$1$  & $160.00000000000000$  & 0\\
$2$  & $-6144.0000000000000$ & $-8192.0000000000000$\\
$3$  & $-138622.00000000000$ & $-18432.000000000000$\\
$4$  & $-524288.00000000000$ & $-622592.00000000000$\\
$5$  & $51735.000000000000$  & $-2048000.0000000000$\\
$6$  & $-14942208.000000000$ & $-10838016.000000000$\\
$7$  & $-23677216.000000000$ & $-23181312.000000000$\\
$8$  & 0                     & $-58982400.000000000$\\
$9$  & $-409311232.00000000$ & $-131383296.00000000$\\
$10$ & $293601280.00000000$  & $-99328000.000000000$\\
$10.000999999999999$ & $-1239859200.0000000$ & $198683136.00000000$ \\
$11$ & $-2502688768.0000000$ & $-561532928.00000000$\\
$12$ & $-12884901888.000000$ & $-875003904.00000000$\\
$13$ & $11492392960.000000$  & $-1385832448.0000000$\\
$14$ & $-12113149952.000000$ & $-1975328768.0000000$\\
$15$ & $30333206528.000000$  & $-3808512000.0000000$\\
$16$ & 0                     & $-6029312000.0000000$\\
$17$ & $370440929280.00000$  & $-9619103744.0000000$\\
$18$ & $-2806761127936.0000$ & $-23619133440.000000$\\
$19$ & $1979979923456.0000$  & $-16210505728.000000$\\
$20$ & $3848290697216.0000$  & $-27193344000.000000$\\\hline
$x$  & Naive Evaluation & Horner Evaluation
 \end{longtable}
\end{center}
This doesn't seem to fulfill the promises made earlier concerning
improved results.
\end{ex}

We can improve the situation by using Horner's algorithm
\emph{with base-points}. We want to rewrite the polynomial to look like
\begin{equation}
  p(z) = c_{0} + (z - b_{0})[c_{1} + (z - b_{1})[c_{2} + \dots (c_{n-1}
      + (z - b_{n-1})c_{n})]].
\end{equation}
At first glance, this seems to make the situation worse. But if we know
the roots of the polynomial, or if we're given the polynomial in this
format\footnote{As will happen with divided differences in polynomial
interpolation.}, then we're in business. If we know the roots, then we
take $c_{n}=1$ and $c_{n-1}=\dots=c_{0}=0$ and the base-points are set
to the roots $b_{k}=r_{k}$.

\begin{algorithm}\label{alg:horner-base-evaluation}
  \caption{Horner polynomial evaluation, with base points $b$}
  \begin{algorithmic}[1]
    \Function{HornerBasePointEval}{$p$,$b$,$x_{0}$} \Comment{$p$ and $b$ are arrays}
      \State$n\gets\Call{PolynomialOrder}{p}$
      \State$r\gets p_{n}$
      \For{$j=1,n$}
        \State $r\gets (r\otimes (x_{0} \ominus b_{n-j})) \oplus p_{n-j}$
      \EndFor
      \State \Return $r$
    \EndFunction
\end{algorithmic}
\end{algorithm}

\begin{lesson}
  We see, in Algorithm~\ref{alg:horner-base-evaluation}, that the length
  of the array $b$ is one less than the length of the array $p$. We
  should begin with an assertion that this is true, if our language
  supports it. In a language like Lisp or \CEE/, which doesn't support
  sophisticated assertion functionality, it suffices to have a macro or
  function which checks if a condition is false (i.e., the assertion
  ``failed''). In which case, some logging message is printed, and the
  program terminates. There should be a short-circuit to ``turn off''
  assertions in production.
\end{lesson}

\begin{chunk}
  The \FORTRAN/ implementation of this:\marginpar{\texttt{polynomial.f90}}
  \lstinputlisting[language=fortran18,firstline=17,lastline=30]{src/polynomial.f90}
\end{chunk}


\begin{ex}\label{ex:polynomial-evaluation:horner:horner-with-base-points-wilkinson}\index{Wilkinson Polynomial}
  Continuing with the Wilkinson polynomial, we can now observe that the
  coefficients $c_{0}=c_{1}=\dots=c_{20}=0$ and the base-points
  $b_{1}=1$, $b_{2}=2$, \dots, $b_{20}=20$. When we use Algorithm~\ref{alg:horner-base-evaluation}
  with these parameters, we have the following result:
  \begin{center}
\begin{longtable}{r|l|l|l}
$x$ & Naive Evaluation & Horner Evaluation & Horner with Base \\\hline
$1$ &$160.00000000000000$ &$0.0000000000000000$ &$0.0000000000000000$ \\
$2$ &$-6144.0000000000000$ &$-8192.0000000000000$ &$0.0000000000000000$ \\
$3$ &$-138622.00000000000$ &$-18432.000000000000$ &$0.0000000000000000$ \\
$4$ &$-524288.00000000000$ &$-622592.00000000000$ &$0.0000000000000000$ \\
$5$ &$51735.000000000000$ &$-2048000.0000000000$ &$0.0000000000000000$ \\
$6$ &$-14942208.000000000$ &$-10838016.000000000$ &$0.0000000000000000$ \\
$7$ &$-23677216.000000000$ &$-23181312.000000000$ &$0.0000000000000000$ \\
$8$ &$0.0000000000000000$ &$-58982400.000000000$ &$0.0000000000000000$ \\
$9$ &$-409311232.00000000$ &$-131383296.00000000$ &$0.0000000000000000$ \\
$10$ &$293601280.00000000$ &$-99328000.000000000$ &$0.0000000000000000$ \\
$10.000999999999999$ &$-1239859200.0000000$ &$198683136.00000000$ &$\mathbf{1316685234.713}1605$\\
$11$ &$-2502688768.0000000$ &$-561532928.00000000$ &$0.0000000000000000$ \\
$12$ &$-12884901888.000000$ &$-875003904.00000000$ &$0.0000000000000000$ \\
$13$ &$11492392960.000000$ &$-1385832448.0000000$ &$0.0000000000000000$ \\
$14$ &$-12113149952.000000$ &$-1975328768.0000000$ &$0.0000000000000000$ \\
$15$ &$30333206528.000000$ &$-3808512000.0000000$ &$0.0000000000000000$ \\
$16$ &$0.0000000000000000$ &$-6029312000.0000000$ &$0.0000000000000000$ \\
$17$ &$370440929280.00000$ &$-9619103744.0000000$ &$0.0000000000000000$ \\
$18$ &$-2806761127936.0000$ &$-23619133440.000000$ &$0.0000000000000000$ \\
$19$ &$1979979923456.0000$ &$-16210505728.000000$ &$0.0000000000000000$ \\
$20$ &$3848290697216.0000$ &$-27193344000.000000$ &$0.0000000000000000$ \\
 \hline
$x$ & Naive Evaluation & Horner Evaluation & Horner with Base
 \end{longtable}
  \end{center}
This is no typo in the right column: Horner's method with offsets
computes Wilkinson's polynomial quite accurately. In fact, the exact
answer for $w_{20}(10.001)$ is
\begin{align}
  w_{20}(10.001) &= \frac{1316685234713890342385967043256460767330153372700300572948849714990001}
    {10^{60}}\nonumber\\
    &\approx 1316685234.713890342.
\end{align}
  Our approximation aggrees with the exact answer to 13 digits. We may
  be more precise, and compute by hand $w_{0}^{*}=w_{20}(x_{0}^{*})$, where
  $x_{0}$ is approximated by the floating-point number
  \begin{equation}
    \begin{split}
      x_{0}^{*} &:= \fl^{-1}(\fl(10.001))\\
      &=\frac{5630062484166541}{562949953421312}.
    \end{split}
  \end{equation}
  We find
  \begin{equation}
    w_{0}^{*} = \frac{n}{d}
  \end{equation}
  where
  \medbreak
  \begingroup
\noindent $n =$ \numprint{13454814548017714270834216765250216272127916488458850895062713092145332371299140852707717077336009753023344177629367925563370707643051703422563451846237105787217147159277944095178850212832025003254893112118372451753978464576605631316434229821111051065036081775579670822271521963775223088547930326974940625}
\endgroup
\medbreak
\noindent and
\medbreak
  \begingroup
\noindent $d =$ \numprint{10218702384817765435680628290748613458265350453429542612493041881278524886369096016818984783322294789577433327842265575649138882500575199542984559607218336872038429045509558663769793133795138494375185186532064890845853749530218856391110938974453986086436459043203870933208875495579361330830770176}
  \endgroup
  \medbreak\noindent%
Or, as a more manageable expression,
  \begin{equation}
    w_{0}^{*} \approx \numprint{1316685234.7131606799}.
  \end{equation}
  Our approximation to this value has an absolute error of approximately
  $1.367\times10^{-7}$ which is nearly as good as it gets (an absolute
  error of $10^{-7}$ in a number $10^{9}$ means we have 16 good digits,
  which exhausts double precision), and a relative error of about
  $1.36657\times10^{-16}$. We could look at the binary form of $w_{0}^{*}$
  and what Horner's method produces, which gives us an even better bound
  on error\footnote{The absolute error is about $1.48315\times 10^{-8}$ and
  relative error about $1.12643\times10^{-17}$.}, but the moral should be understood: Horner's method with
  base points produces optimal results.
\end{ex}
