\section{Chebyshev Interpolation}

We've seen polynomial interpolation approximate complicated functions
like $\sin(x)$. But the approximation left much to be desired. What's
the solution? Adding more observations? Sometimes this causes more
problems than it solves.

\begin{ex}[Runge phenomena]
  Runge studied the exponential function in a clever way. He took
  \begin{equation}
    u := \E^{h}-1
  \end{equation}
  and
  \begin{equation}
    v := x/h
  \end{equation}
  then examined the [truncated] binomial expansion of
  \begin{equation}
    g_{n}(x) = (1 + u)^{v} = 1 + uv + u^{2}\frac{v(v-1)}{1\cdot 2}
    +\dots+u^{n}\frac{v(v-1)(\dots)(v-(n-1))}{1\cdot2\cdot3\cdot(\dots)\cdot n}.
  \end{equation}
  It should converge to $\exp(x)$, right?
  
  Consider the function
  \begin{equation}
    f(x) = \frac{1}{1 + x^{2}}
  \end{equation}
  on the domain $x\in[-1,1]$.
\end{ex}