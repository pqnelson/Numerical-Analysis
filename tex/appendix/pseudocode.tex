\chapter{Pseudocode Conventions}

This is just a grocery list of the conventions adopted for the
pseudocode used in the algorithms:
\begin{itemize}
\item Assignment looks like ``$\langle\mbox{\it var}\rangle\gets\langle\mbox{\it expr}\rangle$''
\item Testing if $x$ is equal to $y$ is done by writing ``$x = y$''.
\item Exceptions are raised with the ``\Fail$\langle$\textit{message}$\rangle$'' keyword
\item Results are returned using the ``\Return$\langle$\textit{expression}$\rangle$'' keyword
\item For-loops used \FORTRAN/-like syntax, ``\algorithmicfor 
  $\langle$\textit{counter variable}$\rangle = \langle$\textit{initial
  value}$\rangle$, $\langle$\textit{final value}$\rangle$'' which
  increments the counter-variable by 1 for each iteration until it is
  greater than the final value
\item While-loops iterate until the test condition fails, or a result is
  returned, or a ``\Break'' is encountered. ``\Break'' corresponds to
  \FORTRAN/ \verb#EXIT#

\end{itemize}