\chapter{Iterative Algorithms Solving Systems of Linear Equations}

\section{Jacobi and Gauss-Seidel}

We ultimately want to construct a sequence of vectors
$\{\vec{x}^{(k)}\}_{k\in\NN}$ which converges to the solution of a
system of linear equations $A\vec{x}=\vec{b}$. The first step will be to
write down some recurrence relation
\begin{equation}
\vec{x}^{(k+1)}=T\vec{x}^{(k)}+\vec{c}.
\end{equation}
How to do this?

We could begin by writing
\begin{equation}
A = L + D + U
\end{equation}
where $L$ is strictly lower-triangular, $D$ is diagonal, and $U$ is
strictly upper-triangular. Then we would take our system of equations
\begin{equation}
(L+D+U)\vec{x}=\vec{b},
\end{equation}
subtract $(L+U)\vec{x}$ from both sides to obtain,
\begin{equation}
D\vec{x}=-(L+U)\vec{x}+\vec{b},
\end{equation}
and finally multiply on the left by $D^{-1}$
\begin{equation}
\vec{x}=-D^{-1}(L+U)\vec{x}+D^{-1}\vec{b}.
\end{equation}
We then use this as our recurrence relation
\begin{equation}\label{eq:iterative-linear:jacobi-matrix-form}
\vec{x}^{(k+1)}=-D^{-1}(L+U)\vec{x}^{(k)}+D^{-1}\vec{b}.
\end{equation}
The presence of an inverse matrix might appear discouraging, but we
should recall the inverse of a diagonal matrix is just the inverse of
its components (which is easily computable!). So, we're good. We
strategy obtained from Eq~\eqref{eq:iterative-linear:jacobi-matrix-form}
is called the \define{Jacobi method}, and its basic structure may be
outlined in Algorithm~\ref{alg:iterative-linear:jacobi-iterative}.

If we don't have any good grounds for picking the initial approximation
$\vec{x}^{(0)}$, then the standard is to pick $\vec{x}^{(0)}=\vec{c}=D^{-1}\vec{b}$.

\begin{algorithm}\label{alg:iterative-linear:jacobi-iterative}
  \caption{Jacobi iterative method}
  \begin{algorithmic}[1]
    \Require $A=(a_{ij})$ is an $n\times n$ matrix
    \Require $a_{ii}\neq0$ for each $i=1,\dots,n$
    \Require $\vec{b}$ is an $n$ vector
    \Ensure $\vec{x}$ is a solution
    \Function{Jacobi}{$A$, $\vec{b}$, $\varepsilon_{\text{tol}}$, $N_{\text{max}}$}
    \State $\vec{p}\gets\vec{0}$ \algorithmiccomment{for previous iteration's guess}
    \For{$k=1$, \dots, $N_{\text{max}}$}
      \For{$i=1$, \dots, $n$}\label{alg:iterative-linear:jacobi-iterative:parallelizable-step}
        \State{$x_{i}\gets (b_{i} - \sum^{n}_{j\neq i}a_{ij}p_{j})/a_{ii}$}
      \EndFor
      \If{$\|\vec{x}-\vec{p}\|<\varepsilon_{\text{tol}}$}
        \State\Return $\vec{x}$
      \EndIf
      \State $\vec{p}\gets\vec{x}$
    \EndFor
    \State\Return $\vec{x}$
  \EndFunction
\end{algorithmic}
\end{algorithm}

\begin{chunk}
  The \FORTRAN/ implementation of Jacobi's method follows the algorithm
  we sketched out. An improved
  implementation would also pivot as needed to avoid any divide-by-zero
  problems. The
  \verb#jacobi_iterate# subroutine will store the approximation in the
  \verb#ans# parameter.\marginpar{\texttt{jacobi.f90}}
  \lstinputlisting[language=fortran18,firstline=27,lastline=57]{src/jacobi.f90}
\end{chunk}

One advantage of the Jacobi method is that we can parallelize the
inner-loop in
Step~\ref{alg:iterative-linear:jacobi-iterative:parallelizable-step}. The
problem is that we typically need many more steps for the Jacobi method
than alternatives.

The astute reader may wonder: why don't we just use the new values of
$x_{i}$ as we compute them? That is, why not compute
\begin{equation}
x_{i}=(b_{i}-\sum^{i-1}_{j=1}a_{ij}x_{j}-\sum^{n}_{j=i+1}a_{ij}p_{j})/a_{ii},
\end{equation}
instead of reusing the previous iteration's approximation throughout?
This is precisely the \define{Gauss-Seidel method}. Its skeleton may be
found in Algorithm~\ref{alg:iterative-linear:gauss-seidel}.

\begin{algorithm}\label{alg:iterative-linear:gauss-seidel}
  \caption{Gauss-Seidel iterative method}
  \begin{algorithmic}[1]
    \Require $A=(a_{ij})$ is an $n\times n$ matrix
    \Require $a_{ii}\neq0$ for each $i=1,\dots,n$
    \Require $\vec{b}$ is an $n$ vector
    \Ensure $\vec{x}$ is a solution
    \Function{Gauss-Seidel}{$A$, $\vec{b}$, $\varepsilon_{\text{tol}}$, $N_{\text{max}}$}
    \State $\vec{p}\gets\vec{0}$ \algorithmiccomment{for previous iteration's guess}
    \For{$k=1$, \dots, $N_{\text{max}}$}
      \For{$i=1$,\dots, $n$}
        \State{$x_{i}\gets (b_{i} - \sum^{i-1}_{j=1}a_{ij}x_{j}- \sum^{n}_{j=i+1}a_{ij}p_{j})/a_{ii}$}
      \EndFor
      \If{$\|\vec{x}-\vec{p}\|<\varepsilon_{\text{tol}}$}
        \State\Return $\vec{x}$
      \EndIf
      \State $\vec{p}\gets\vec{x}$
    \EndFor
    \State\Return $\vec{x}$
  \EndFunction
\end{algorithmic}
\end{algorithm}

\begin{chunk}
  The \FORTRAN/ implementation of Gauss-Seidel is straightforward. The
  only subtle bit is that at the start of each iteration, we don't need
  to initialize
  $\vec{x}^{(k+1)}=\vec{x}^{(k)}$ since it's already storing the
  previous iteration's results. So we can update the values
  in-place. Special attention must be paid that we are subtracting from
  $b_{i}$ the values $a_{ij}x_{j}$ for all $j\neq i$, and that this is
  stored in $x_{i}$. We still need \verb#prev_iter# for the stopping
  condition, which is why we keep it around.
  The same caveat about pivoting mentioned regarding
  the Jacobi implementation applies here as well. The
  \verb#gauss_seidel# subroutine will store the approximation in the
  \verb#ans# parameter.\marginpar{\texttt{jacobi.f90}}
  \lstinputlisting[language=fortran18,firstline=59,lastline=90]{src/jacobi.f90}
\end{chunk}

We have some estimate on the error bounds using the following result.

\begin{defn}\index{Spectral Radius}
Let $T$ be an $n\times n$ matrix over the reals\footnote{We could extend
this to the complex numbers, or restrict it to the rationals, or any
other subfield of $\CC$ if you want. But floating point numbers are
supposed to approximate $\RR$, so we are really interested in that field.}.
We define the \define{Spectral Radius} of $T$ to be the non-negative
real number $\rho(T) := \max\spec(T)$ equal to the maximum of the
absolute value of the eigenvalues of $T$.
\end{defn}

\begin{lemma}
The solution to $\vec{x}=T\vec{x}+\vec{c}$ is $\vec{x}=(I-T)^{-1}\vec{c}$.
Moreover the sequence $\{\vec{x}^{(k)}\}_{k\in\NN}$ defined by
$\vec{x}^{(k+1)}=T\vec{x}^{(k)}+\vec{c}$ converges to the solution if
and only if $\rho(T)<1$.
\end{lemma}

\begin{proof}[Proof sketch]
$(\Longleftarrow)$ Assume $\rho(T)<1$. We want to prove
convergence. Observe that 
\begin{align*}
\vec{x}^{(k)}
&=T\vec{x}^{(k-1)}+\vec{c}\\
&=T(T\vec{x}^{(k-2)}+\vec{c})+\vec{c}=T^{2}\vec{x}^{(k-2)}+(I+T)\vec{c}\\
&=\dots=T^{k}\vec{x}^{(0)}+(I+T+\dots+T^{k-1})\vec{c}.
\end{align*}
Taking the $k\to\infty$ limit gives us
\begin{equation}
\lim_{k\to\infty}\vec{x}^{(k)}=(I-T)^{-1}\vec{c},
\end{equation}
since $T^{k}\to0$ as $k\to\infty$. This gives us convergence.

$(\Longrightarrow)$ Assume that $\vec{x}^{(k)}\to\vec{x}$ for any
initial value $\vec{x}^{(0)}$ and $\vec{x}=T\vec{x}+\vec{c}$. Now let us
have $\vec{y}$ be any arbitrary value, and let us have
$\vec{x}^{(0)}=\vec{x}-\vec{y}$ be the initial value for the sequence.
Observe
\begin{equation}
\begin{split}
\vec{x}-\vec{x}^{(k)}&=(T\vec{x}+\vec{c})-(T\vec{x}^{(k-1)}+\vec{c})\\
&=T(\vec{x}-\vec{x}^{(k-1)}),
\end{split}
\end{equation}
so
\begin{equation}
\vec{x}-\vec{x}^{(k)}=T(\vec{x}-\vec{x}^{(k-1)})=T^{2}(\vec{x}-\vec{x}^{(k-2)})
=\dots=T^{k}(\vec{x}-\vec{x}^{(0)})=T^{k}\vec{y}.
\end{equation}
In particular, since we assumed the sequence $\vec{x}^{(k)}$ converges,
this means
\begin{equation}
\lim_{k\to\infty}T^{k}\vec{y}=\vec{0},
\end{equation}
which implies $\rho(T)<1$.
\end{proof}

\begin{prop}
If $\|T\|<1$ for any induced matrix norm and $\vec{c}$ is any given
vector, then the sequence $\{\vec{x}^{(k)}\}_{k\in\NN}$ converges to the
solution $\vec{x}$ for any initial guess $\vec{x}^{(0)}$. Furthermore,
we have the following error bounds:
\begin{enumerate}
\item $\|\vec{x}-\vec{x}^{(k)}\|\leq\|T\|^{k}\|\vec{x}^{(0)}-\vec{x}\|$;
\item $\displaystyle\|\vec{x}-\vec{x}^{(k)}\|\leq\frac{\|T\|^{k}}{1-\|T\|}\|\vec{x}^{(1)}-\vec{x}^{(0)}\|$.
\end{enumerate}
\end{prop}

\begin{proof}
Observe that
\begin{subequations}
\begin{equation}
\vec{x}=T\vec{x}+\vec{c}
\end{equation}
and
\begin{equation}
\vec{x}^{(k)}=T\vec{x}^{(k-1)}+\vec{c}.
\end{equation}
\end{subequations}
Therefore
\begin{equation}
\vec{x}-\vec{x}^{(k)}=(T\vec{x}+\vec{c})-(T\vec{x}^{(k-1)}+\vec{c})=T(\vec{x}-\vec{x}^{(k-1)}).
\end{equation}
Therefore
\begin{equation}
\vec{x}-\vec{x}^{(k)}=T^{k}(\vec{x}-\vec{x}^{(0)}).
\end{equation}
Taking the norm on both sides produces the first error bound.

Observe that
\begin{equation}
\begin{split}
\vec{x}^{(k+1)}-\vec{x}^{(k)}
&=(T\vec{x}^{(k)}+\vec{c})-(T\vec{x}^{(k-1)}+\vec{c})\\
&=T(\vec{x}^{(k)}-\vec{x}^{(k-1)}).
\end{split}
\end{equation}
Then by induction, we would have
\begin{equation}
\vec{x}^{(k+1)}-\vec{x}^{(k)}=T^{k}(\vec{x}^{(1)}-\vec{x}^{(0)}).
\end{equation}
So far, so good. Now, for any nonzero $m\in\NN$, $m\neq0$, we would have
\begin{align*}
\|\vec{x}^{(k+m)}-\vec{x}^{(k)}\|&=
\|\vec{x}^{(k+m)}-\vec{x}^{(k+m-1)}+\vec{x}^{(k+m-1)}-\dots+\vec{x}^{(k+1)}-\vec{x}^{(k)}\|\\
&\leq\|\vec{x}^{(k+m)}-\vec{x}^{(k+m-1)}\|+\|\vec{x}^{(k+m-1)}-\vec{x}^{(k+m-2)}\|+\dots+\|\vec{x}^{(k+1)}-\vec{x}^{(k)}\|\\
&\leq \|T^{k+m-1}\|\cdot\|\vec{x}^{(1)}-\vec{x}^{(0)}\|+\|T^{k+m-2}\|\cdot\|\vec{x}^{(1)}-\vec{x}^{(0)}\|+\dots+\|T^{k}\|\cdot\|\vec{x}^{(1)}-\vec{x}^{(0)}\|.
\end{align*}
Taking the $m\to\infty$ limit yields the result
\begin{equation}
  \begin{split}
\|\vec{x}-\vec{x}^{(k)}\|&\leq\|T^{k}\sum^{\infty}_{n=0}T^{n}\|\cdot\|\vec{x}^{(1)}-\vec{x}^{(0)}\|\\
&\leq\|T^{k}\|\left(\sum^{\infty}_{n=0}\|T^{n}\|\right)\|\vec{x}^{(1)}-\vec{x}^{(0)}\|=\frac{\|T^{k}\|}{1-\|T\|}\|\vec{x}^{(1)}-\vec{x}^{(0)}\|.
  \end{split}
\end{equation}
Note that submultiplicativity of the matrix norm is key in obtaining the
second line of the previous equation. Thus we obtain the second error bound.
\end{proof}

\section{Successive Over-Relaxation}

Now, Gauss-Seidel gives us the iterative method:
\begin{equation}
a_{ii}x_{i}^{(k)}=\underbrace{b_{i}-\sum^{i-1}_{j=1}a_{ij}x^{(k)}_{j}-\sum^{n}_{j=i}a_{ij}x_{j}^{(k-1)}}_{\mbox{looks like a residual}}
+a_{ii}x_{i}^{(k-1)}.
\end{equation}
We shall introduce some notation to assist our discussion.

Let us denote by $\vec{x}_{i}^{(k)}$ the $i^{\text{th}}$ step in
computing the $k^{\text{th}}$ iteratative approximation in Gauss-Seidel,
i.e.,
\begin{equation}
\vec{x}_{i}^{(k)}=(x_{1}^{(k)}, x_{2}^{(k)},\dots, x_{i-1}^{(k)},
x_{i}^{(k-1)}, \dots, x_{n}^{(k-1)}).
\end{equation}
Let us denote the residual for this vector be denoted by
\begin{equation}
\vec{r}_{i}^{(k)}=\vec{b}-A\vec{x}_{i}^{(k)}.
\end{equation}

Now, using this notation, we can write Gauss-Seidel as
\begin{equation}
a_{ii}x^{(k)}_{i} = a_{ii}x^{(k-1)}_{i} + r_{ii}^{(k)},
\end{equation}
equivalently
\begin{equation}
x^{(k)}_{i} = x^{(k-1)}_{i} + \frac{r_{ii}^{(k)}}{a_{ii}}.
\end{equation}
So far, we have just described Gauss-Seidel with different notation.

Successive over-relaxation introduces a new parameter $\omega>0$ and
modified Gauss-Seidel by
\begin{equation}
x^{(k)}_{i} = x^{(k-1)}_{i} + \omega\frac{r_{ii}^{(k)}}{a_{ii}}.
\end{equation}
When $\omega=1$ we recover Gauss-Seidel, when $\omega<1$ we have
under-relaxation (useful for obtaining convergence for a divergent
situation), and when $\omega>1$ we have over-relaxation. The problem
before us now is to pick an ``optimal'' $\omega$.

\begin{algorithm}\label{alg:iterative-linear:sor:sor}
  \caption{Successive over-relaxation}
  \begin{algorithmic}[1]
    \Require $A=(a_{ij})$ is an $n\times n$ matrix
    \Require $a_{ii}\neq0$ for each $i=1,\dots,n$
    \Require $0<\omega$
    \Require $\vec{b}$ is an $n$ vector
    \Ensure $\vec{x}$ is a solution
    \Function{Successive Over-Relaxation}{$A$, $\vec{b}$, $\vec{x}_{\text{init}}$, $\omega$, $\varepsilon_{\text{tol}}$, $N_{\text{max}}$}
    \State $\vec{p}\gets\vec{x}_{\text{init}}$ \algorithmiccomment{for previous iteration's guess}
    \For{$k=1$, \dots, $N_{\text{max}}$}
      \For{$i=1$, \dots, $n$}
        \State{$x_{i}\gets (1-\omega)p_{i} + \omega (b_{i} - \sum^{i-1}_{j=1}a_{ij}x_{j}- \sum^{n}_{j=i+1}a_{ij}p_{j})/a_{ii}$}
      \EndFor
      \If{$\|\vec{x}-\vec{p}\|<\varepsilon_{\text{tol}}$}
        \State\Return $\vec{x}$
      \EndIf
      \State $\vec{p}\gets\vec{x}$
    \EndFor
    \State\Return $\vec{x}$
  \EndFunction
\end{algorithmic}
\end{algorithm}

Before we analyze the convergence properties of successive
over-relaxation, let us review a few definitions from linear algebra.

\begin{definition}
We call an $n\times n$ matrix $A$ \define{Symmetric} if $\transpose{A}=A$,
and \define{Antisymmetric} (or \emph{skew-symmetric}) if $\transpose{A}=-A$.
\end{definition}

\begin{prop}
Let $A$ be an antisymmetric $n\times n$ matrix. Then for any
$\vec{x}\in\RR^{n}$ we have $\transpose{\vec{x}}A\vec{x}=0$.
\end{prop}

\begin{prop}
Let $M$ be any $n\times n$ real matrix. We can write $M$ uniquely as the
sum of a symmetric matrix $S$ and antisymmetric matrix $A$, i.e.,
$M=A+S$ where $A=(M-\transpose{M})/2$ and $S=(M+\transpose{M})/2$.
\end{prop}

\begin{definition}
Let $A$ be a symmetric $n\times n$ real matrix. We call $A$
\define{Positive-Definite} if for all (nonzero)
$\vec{x}\in\RR^{n}\setminus\{\vec{0}\}$ we have
$\transpose{\vec{x}}A\vec{x}>0$.
\end{definition}

\begin{rmk}
Note that, by the previous propositions in this section, we could
ostensibly weaken the notion of a positive-definite matrix to any
nonsingular matrix $M$ since
\begin{equation}
  \begin{split}
\transpose{\vec{x}}M\vec{x}&=\transpose{\vec{x}}S\vec{x}+\transpose{\vec{x}}A\vec{x}\\
&=\transpose{\vec{x}}S\vec{x},
  \end{split}
\end{equation}
where $A=(M-\transpose{M})/2$ and $S=(M+\transpose{M})/2$.
In practice, however, we must be careful because floating-point arithmetic
doesn't necessarily respect $\transpose{\vec{x}}A\vec{x}=0$.
\end{rmk}

\begin{theorem}[Kahan]
If $a_{ii}\neq0$ for each $i=1,\dots,n$, then
$\rho(T_{\omega})\geq|\omega-1|$ where if $A=D+L+U$ we have
\[ T_{\omega} = (D+\omega L)^{-1}[(1-\omega)D-\omega U]. \]
\end{theorem}

This was part of Kahan's 1958 doctoral thesis.

\begin{theorem}[{Ostrowski--Reich~\cite{ostrowski1954linear,reich1949convergence}}]
If $A$ is a [symmetric] positive-definite matrix and $0<\omega<2$, then
successive over-relaxation converges for any choice of initial $\vec{x}^{(0)}$.
\end{theorem}

\begin{theorem}
Let $T_{g}$ and $T_{j}$ be the iterative matrices associated with the
Gauss-Seidel and Jacobi methods, respectively. If $A$ is symmetric
positive-definite, then $\rho(T_{g})=[\rho(T_{j})]^{2}<1$ and the
optimal choice of $\omega$ for successive over-relaxation is
\begin{equation}\label{eq:iterative:sor:optimal-w-for-positive-definite-mat}
\omega = \frac{2}{1 + \sqrt{1 - [\rho(T_{j})]^{2}}}.
\end{equation}
With this choice, $\rho(T_{\omega})=\omega-1$.
\end{theorem}

\begin{rmk}
When $A=D+L+U$ satisfies $\det(\lambda D+zL+z^{-1}U)=\det(\lambda D+L+U)$
for any $z\in\CC\setminus\{0\}$ and $\lambda\in\CC$, then Eq~\eqref{eq:iterative:sor:optimal-w-for-positive-definite-mat}
gives the optimal $\omega$. This is the more general situation, since it
automatically describes tridiagonal matrices. The more general situation
was proven in Young's doctoral thesis~\cite{young1950} and may be found
in \S10.1 of Greenbaum~\cite{greenbaum1997iterative} as well as \S4.6.2 of Hackbusch~\cite{hackbusch2016iterative}.
\end{rmk}


\section{Conjugate Gradient Method}

The ``trick'' to the conjugate gradient method is a standard one
deployed by mathematicians: change the problem to one more palatable to
a different toolkit. However, the real intuition is to use
Graham-Schmidt on the residual vectors in an iterative procedure.

\begin{theorem}
Let $A$ be a positive definite matrix, and $\vec{b}$ be a vector.
Then $\vec{x}_{*}$ solves $A\vec{x}=\vec{b}$ if and only if
$\vec{x}_{*}$ minimizes
\begin{equation}
g(\vec{x})=\langle\vec{x},A\vec{x}\rangle-2\langle\vec{x},\vec{b}\rangle.
\end{equation}
\end{theorem}

\begin{proof}
Obvious.
\end{proof}

Let $\vec{x}^{(0)}$ be an initial guess for $\vec{x}_{*}$. If
$A\vec{x}^{(0)}=\vec{b}$, then we're done. We'll assume this is not the
case. Let
$\vec{v}^{(1)}\neq\vec{0}$ be an initial search direction. Compute
\begin{equation}
t_{1} = \frac{\langle\vec{v}^{(1)},\vec{b}-A\vec{x}^{(0)}\rangle}{\langle\vec{v}^{(1)},A\vec{v}^{(1)}\rangle},
\end{equation}
then set
\begin{equation}
\vec{x}^{(1)} = \vec{x}^{(0)}+t_{1}\vec{v}^{(1)}.
\end{equation}
Now, look, $t_{1}$ describes the overlap of the residual vector
$\vec{r}^{(0)}=\vec{b}-A\vec{x}^{(0)}$ in the direction of
$\vec{v}^{(1)}$, so what $\vec{x}^{(1)}$ computes is the correction in
the $\vec{v}^{(1)}$ direction to minimize the residual vector's
$\vec{v}^{(1)}$ component. This leads to
$g(\vec{x}^{(1)})<g(\vec{x}^{(0)})$.

If we had access to $n$ direction vectors $\vec{v}^{(1)}$, \dots,
$\vec{v}^{(n)}$ which are $A$-orthogonal,
\begin{equation}
\langle\vec{v}^{(i)},A\vec{v}^{(j)}\rangle=0\quad\mbox{if }i\neq j,
\end{equation}
then we can iterative step by taking
\begin{equation}
\vec{x}^{(k)} = \vec{x}^{(k-1)} + t_{k}\vec{v}^{(k)},
\end{equation}
where
\begin{equation}
t_{k} = \frac{\langle\vec{v}^{(k)},\vec{r}^{(k-1)}\rangle}{\langle\vec{v}^{(k)},A\vec{v}^{(k)}\rangle}.
\end{equation}
Since we keep moving towards the direction which systematically reduces
the residual vector, one dimension at a time, this leads us to conclude
$g(\vec{x}^{(k)})<g(\vec{x}^{(k-1)})$. 
The problem before us now is how to pick the $A$-orthogonal direction
vectors $\vec{v}^{(k)}$?

We see that
\begin{equation}
\langle\vec{r}^{(k-1)},\vec{v}^{(i)}\rangle=0
\end{equation}
for $i=1,$ $2$, \dots, $k-1$. Then we can use the residual vector
$\vec{r}^{(k-1)}$ to determine $\vec{v}^{(k)}$ by
\begin{subequations}
\begin{equation}
\vec{v}^{(k)}=\vec{r}^{(k-1)}+s_{k-1}\vec{v}^{(k-1)},
\end{equation}
and now we just need to determine what $s_{k-1}$. Now we just need to
pick $s_{k-1}$ to make $\vec{v}^{(k)}$ ``$A$-orthogonal'' to
$\vec{v}^{(k-1)}$,
\begin{equation}
\langle\vec{v}^{(k-1)},A\vec{v}^{(k)}\rangle=0.
\end{equation}
Since
\begin{equation}
A\vec{v}^{(k)}=A\vec{r}^{(k-1)}+s_{k-1}A\vec{v}^{(k-1)},
\end{equation}
and
\begin{equation}
0=\langle\vec{v}^{(k-1)},A\vec{v}^{(k)}\rangle=
\langle\vec{v}^{(k-1)},A\vec{r}^{(k-1)}\rangle
+s_{k-1}\langle\vec{v}^{(k-1)},A\vec{v}^{(k-1)}\rangle,
\end{equation}
we can solve this to find
\begin{equation}\label{eq:iterative:conjugate-gradient:sk-init}
s_{k-1} = -\frac{\langle\vec{v}^{(k-1)},A\vec{r}^{(k-1)}\rangle}{\langle\vec{v}^{(k-1)},A\vec{v}^{(k-1)}\rangle}.
\end{equation}
\end{subequations}
It can be shown that $\langle\vec{v}^{(k)},A\vec{v}^{(j)}\rangle=0$ for
$i=1$, \dots, $k-1$.

The goal now will be to express $s_{k}$ and $t_{k}$ using the residual
vectors $\vec{r}^{(k)}$ and $\vec{r}^{(k-1)}$.

We see that, by direct calculation,
\begin{subequations}
\begin{align}
t_{k} &= \frac{\langle\vec{v}^{(k)},\vec{r}^{(k-1)}\rangle}{\langle\vec{v}^{(k)},A\vec{v}^{(k)}\rangle}\\
&= \frac{\langle\vec{r}^{(k-1)}+s_{k-1}\vec{v}^{(k-1)},\vec{r}^{(k-1)}\rangle}{\langle\vec{v}^{(k)},A\vec{v}^{(k)}\rangle}\\
&= \frac{\langle\vec{r}^{(k-1)},\vec{r}^{(k-1)}\rangle}{\langle\vec{v}^{(k)},A\vec{v}^{(k)}\rangle}
+ s_{k-1}\frac{\langle\vec{v}^{(k-1)},\vec{r}^{(k-1)}\rangle}{\langle\vec{v}^{(k)},A\vec{v}^{(k)}\rangle}
\end{align}
\end{subequations}
Since $\langle\vec{v}^{(k-1)},\vec{r}^{(k-1)}\rangle=0$, we conclude
\begin{equation}\label{eq:iterative:conjugate-gradient:tk}
\boxed{t_{k} = \frac{\langle\vec{r}^{(k-1)},\vec{r}^{(k-1)}\rangle}{\langle\vec{v}^{(k)},A\vec{v}^{(k)}\rangle}.} 
\end{equation}
Great, this allows us to use the residual vectors to determine the $t_{k}$.

Now, we will determine $s_{k}$ using residual vectors, and this will
enable us to find the direction vectors. We begin by remembering
\begin{subequations}
\begin{equation}
\vec{x}^{(k)}=\vec{x}^{(k-1)}+t_{k}\vec{v}^{(k)}.
\end{equation}
Then we compute $\vec{r}^{(k)}$ the residual vector by multiplying both
sides on the left by $A$ and subtracting out by $\vec{b}$,
\begin{equation}
A\vec{x}^{(k)}-\vec{b}=A\vec{x}^{(k-1)}-\vec{b}+t_{k}A\vec{v}^{(k)},
\end{equation}
hence
\begin{equation}
\vec{r}^{(k)}=\vec{r}^{(k-1)}+t_{k}A\vec{v}^{(k)}.
\end{equation}
We have
\begin{equation}
  \langle\vec{r}^{(k)},\vec{r}^{(k)}\rangle
  =\langle\vec{r}^{(k-1)}-t_{k}A\vec{v}^{(k)},\vec{r}^{(k)}\rangle
  =-t_{k}\langle A\vec{v}^{(k)},\vec{r}^{(k)}\rangle.
\end{equation}
Now we use Eq~\eqref{eq:iterative:conjugate-gradient:tk} to give us
another relationship,
\begin{equation}
\langle\vec{r}^{(k-1)},\vec{r}^{(k-1)}\rangle=t_{k}\langle\vec{v}^{(k)},A\vec{v}^{(k)}\rangle.
\end{equation}
Then we have, starting with Eq~\eqref{eq:iterative:conjugate-gradient:sk-init},
\begin{align}
s_{k}
&=-\frac{\langle\vec{v}^{(k)},A\vec{r}^{(k)}\rangle}{\langle\vec{v}^{(k)},A\vec{v}^{(k)}\rangle}\\
&=-\frac{\langle\vec{r}^{(k)},A\vec{v}^{(k)}\rangle}{\langle\vec{v}^{(k)},A\vec{v}^{(k)}\rangle}\\
&=\frac{(1/t_{k})\langle\vec{r}^{(k)},A\vec{r}^{(k)}\rangle}{(1/t_{k})\langle\vec{r}^{(k-1)},\vec{r}^{(k-1)}\rangle}\\
&=\frac{\langle\vec{r}^{(k)},A\vec{r}^{(k)}\rangle}{\langle\vec{r}^{(k-1)},\vec{r}^{(k-1)}\rangle}.
\end{align}
\end{subequations}
This culminates our derivation of the conjugate gradient method.

In summary, we have the conjugate gradient method be given by the
following equations:
\begin{align}
t_{k} &= \frac{\langle\vec{r}^{(k-1)},\vec{r}^{(k-1)}\rangle}{\langle\vec{v}^{(k)},A\vec{v}^{(k)}\rangle}\\
\vec{x}^{(k)} &= \vec{x}^{(k-1)} + t_{k}\vec{v}^{(k)}\\
\vec{r}^{(k)} &= \vec{r}^{(k-1)} - t_{k}A\vec{v}^{(k)}\\
s_{k} &= \frac{\langle\vec{r}^{(k)},A\vec{r}^{(k)}\rangle}{\langle\vec{r}^{(k-1)},\vec{r}^{(k-1)}\rangle}\\
\vec{v}^{(k+1)} &= \vec{r}^{(k)} + s_{k}\vec{v}^{(k)}.
\end{align}

\subsection{Conditioned Conjugate Gradient Method}

If the matrix $A$ is ill-conditioned, then the conjugate gradient method
is highly susceptible to rounding errors. The trick is to select a
nonsingular ``conditioning matrix'' $C$ such that
\begin{equation}
\widetilde{A}=C^{-1}A\transpose{(C^{-1})}
\end{equation}
is better conditioned. Then our system of equations we would want to
solve would be
\begin{subequations}
\begin{equation}
\widetilde{A}\widetilde{\vec{x}}=\widetilde{\vec{b}}
\end{equation}
or
\begin{equation}
(C^{-1}A\transpose{(C^{-1})})(\transpose{C}\vec{x})=C^{-1}\vec{b}.
\end{equation}
\end{subequations}
Usually it's desirable to take $C$ to be a diagonal matrix, e.g., if
$A=D+L+U$ has all nonzero diagonal entries, then take the component-wise
inverse squareroot of the absolute value of the diagonal entries
$C=|D|^{-1/2}$.

\textbf{However,} instead of solving for $\widetilde{\vec{x}}$ and then
multiplying by $\transpose{(C^{-1})}$, we can incorporate the
conditioning \emph{into} the conjugate gradient algorithm
\emph{implicitly}.

We find by direct computation and recalling
$\widetilde{\vec{x}}^{(k)}=\transpose{C}\vec{x}^{(k)}$:
\begin{calculation}
  \widetilde{\vec{r}}^{(k)}
\step{definition of residual}
  \widetilde{\vec{b}} - \widetilde{A}\widetilde{\vec{x}}^{(k)}
\step{unfolding tilde notation}
  C^{-1}\vec{b}-(C^{-1}A\transpose{(C^{-1})})(\transpose{C}\vec{x}^{(k)})
\step{since $I=\transpose{(C^{-1})}\transpose{C}$, distributivity}
  C^{-1}(\vec{b}-A\vec{x}^{(k)})
\step{definition of residual}
  C^{-1}\vec{r}^{(k)}.
\end{calculation}
We therefore denote the residual vector for the conditioned system
$\vec{w}^{(k)}=C^{-1}\vec{r}^{(k)}$. \emph{This will need to be computed for each step.}

We want to determine the scalar $\widetilde{s}_{k}$ to pick the next
direction vector $\widetilde{\vec{v}}^{(k)} = \transpose{C}\vec{v}^{(k)}$.
As before,
\begin{equation}
\boxed{\widetilde{s}_{k} = \frac{\langle\vec{w}^{(k)},\vec{w}^{(k)}\rangle}{\langle\vec{w}^{(k-1)},\vec{w}^{(k-1)}\rangle}}
\end{equation}
where $\vec{w}$ is the residual vector.

The scalar $\widetilde{t}_{k}$ is computed similarly, if we just naively
put tildes on things we would expect
\begin{calculation}
  \widetilde{t}_{k}
\step{unfolding $t_{k}$}
  \frac{\langle\widetilde{\vec{r}}^{(k-1)},\widetilde{\vec{r}}^{(k-1)}\rangle}{\langle\widetilde{\vec{v}}^{(k)},A\widetilde{\vec{v}}^{(k)}\rangle}
\step{unfolding tildes}
  \frac{\langle C^{-1}\vec{r}^{(k-1)}, C^{-1}\vec{r}^{(k-1)}\rangle}%
     {\langle\transpose{C}\vec{v}^{(k)},C^{-1}A(\transpose{C})^{-1}\transpose{C}\widetilde{v}^{(k)}\rangle}
\step{since $(\transpose{C})^{-1}\transpose{C}=I$, associativity}
  \frac{\langle C^{-1}\vec{r}^{(k-1)}, C^{-1}\vec{r}^{(k-1)}\rangle}%
     {\langle\transpose{C}\vec{v}^{(k)},C^{-1}A\vec{v}^{(k)}\rangle}
\step{folding in $\vec{w}^{(k-1)}$}
  \frac{\langle \vec{w}^{(k-1)}, \vec{w}^{(k-1)}\rangle}%
     {\langle\transpose{C}\vec{v}^{(k)},C^{-1}A\vec{v}^{(k)}\rangle}
\step{since $C$ is real and symmetric}
  \frac{\langle \vec{w}^{(k-1)}, \vec{w}^{(k-1)}\rangle}%
     {\langle\vec{v}^{(k)},CC^{-1}A\vec{v}^{(k)}\rangle}
\step{since $CC^{-1}=I$}
  \frac{\langle \vec{w}^{(k-1)}, \vec{w}^{(k-1)}\rangle}%
     {\langle\vec{v}^{(k)},A\vec{v}^{(k)}\rangle}
\end{calculation}
thus giving us
\begin{equation}
\boxed{\widetilde{t}_{k} = \frac{\langle \vec{w}^{(k-1)}, \vec{w}^{(k-1)}\rangle}%
     {\langle\vec{v}^{(k)},A\widetilde{v}^{(k)}\rangle}.}
\end{equation}
Similarly, from
\begin{equation*}
\widetilde{\vec{x}}^{(k)}=\widetilde{\vec{x}}^{(k-1)} + \widetilde{t}_{k}\widetilde{\vec{v}}^{(k)},
\end{equation*}
we have
\begin{equation*}
\transpose{C}\vec{x}^{(k)}=\transpose{C}\vec{x}^{(k-1)} + \widetilde{t}_{k}\transpose{C}\vec{v}^{(k)},
\end{equation*}
and therefore multiplying on both side from the left by $(\transpose{C})^{-1}$,
\begin{equation}
\boxed{\vec{x}^{(k)}=\vec{x}^{(k-1)} + \widetilde{t}_{k}\vec{v}^{(k)}.}
\end{equation}
Similarly, we can find
\begin{calculation}
  \widetilde{\vec{r}}^{(k)} = \widetilde{\vec{r}}^{(k-1)} - \widetilde{t}_{k}\widetilde{A}\widetilde{\vec{v}}^{(k)}
\step[\equiv]{unfolding tildes}
  C^{-1}\vec{r}^{(k)} = C^{-1}\vec{r}^{(k-1)} - \widetilde{t}_{k} C^{-1}A\transpose{(C^{-1})}\transpose{C}\vec{v}^{(k)}
\step[\equiv]{multiply through by $C$ on the left}
  \vec{r}^{(k)} = \vec{r}^{(k-1)} - \widetilde{t}_{k} A\transpose{(C^{-1})}\transpose{C}\vec{v}^{(k)}
\step[\equiv]{since $I=\transpose{(C^{-1})}\transpose{C}$}
  \vec{r}^{(k)} = \vec{r}^{(k-1)} - \widetilde{t}_{k} A\vec{v}^{(k)}
\end{calculation}
Hence
\begin{equation}
\boxed{\vec{r}^{(k)} = \vec{r}^{(k-1)} - \widetilde{t}_{k} A\vec{v}^{(k)}.}
\end{equation}
Finally, we find the next direction vector $\vec{v}^{(k+1)}$, again by
direct calculation
\begin{calculation}
  \widetilde{\vec{v}}^{(k+1)}=\widetilde{\vec{r}}^{(k)} + \widetilde{s}_{k}\widetilde{\vec{v}}^{(k)}
\step[\equiv]{unfolding tildes}
  \transpose{C}\vec{v}^{(k+1)}=C^{-1}\vec{r}^{(k)} + \widetilde{s}_{k}\transpose{C}\vec{v}^{(k)}
\step[\equiv]{folding in $\vec{w}^{(k)}$}
  \transpose{C}\vec{v}^{(k+1)}=\vec{w}^{(k)} + \widetilde{s}_{k}\transpose{C}\vec{v}^{(k)}
\step[\equiv]{multiply both sides by $(\transpose{C})^{-1}$ on the left}
  \vec{v}^{(k+1)}=(\transpose{C})^{-1}\vec{w}^{(k)} + \widetilde{s}_{k}\vec{v}^{(k)},
\end{calculation}
which gives us:
\begin{equation}
\boxed{\vec{v}^{(k+1)}=(\transpose{C})^{-1}\vec{w}^{(k)} + \widetilde{s}_{k}\vec{v}^{(k)}.}
\end{equation}
These boxed equations are precisely the conjugate-gradient method
applied to the conditioned system. Let us now collate these results in
one location:
\begin{align*}
\widetilde{s}_{k} &= \frac{\langle\vec{w}^{(k)},\vec{w}^{(k)}\rangle}{\langle\vec{w}^{(k-1)},\vec{w}^{(k-1)}\rangle}\\
\widetilde{t}_{k} &= \frac{\langle \vec{w}^{(k-1)}, \vec{w}^{(k-1)}\rangle}%
     {\langle\vec{v}^{(k)},A\widetilde{v}^{(k)}\rangle}\\
\vec{x}^{(k)} &= \vec{x}^{(k-1)} + \widetilde{t}_{k}\vec{v}^{(k)}\\
\vec{r}^{(k)} &= \vec{r}^{(k-1)} - \widetilde{t}_{k} A\vec{v}^{(k)}\\
\vec{w}^{(k)} &= C^{-1}\vec{r}^{(k)}\\
\vec{v}^{(k+1)} &= (\transpose{C})^{-1}\vec{w}^{(k)} + \widetilde{s}_{k}\vec{v}^{(k)}
\end{align*}
It's usually best to pick some $C$ which is easy to invert and take the
transpose of (hint: diagonal matrices).

\begin{algorithm}\label{alg:iterative-linear:conjugate-gradient}
  \caption{Preconditioned conjugate gradient method}
  \begin{algorithmic}[1]
    \Require $A=(a_{ij})$ is an $n\times n$ positive-definite matrix
    \Require $a_{ii}\neq0$ for each $i=1,\dots,n$
    \Require $C^{-1}=(c_{ij})$ is an $n\times n$ matrix
    \Require $\vec{b}$ is an $n$ vector
    \Require $\vec{x}^{(0)}$ is some initial guess $n$ vector
    \Ensure $\vec{x}$ is a solution
    \Function{ConjugateGradient}{$A$, $C^{-1}$, $\vec{b}$, $\vec{x}_{\text{init}}$, $\omega$, $\varepsilon_{\text{tol}}$, $N_{\text{max}}$}
    \State $\vec{x}\gets\vec{x}^{(0)}$ \algorithmiccomment{initialize}
    \State $\vec{r}\gets \vec{b}-A\vec{x}$
    \State $\vec{w}\gets C^{-1}\vec{r}$
    \State $\vec{v}\gets\transpose{(C^{-1})}\vec{w}$
    \State $\alpha\gets\sum^{n}_{j=1}w_{j}^{2}$

    \For{$k=1$, \dots, $N_{\text{max}}$}
      \If{$\|\vec{v}\|<\varepsilon_{\text{tol}}$}
        \State\Return $\vec{x}$
      \EndIf
      \State $\vec{u}\gets A\vec{v}$
      \State $t\gets \alpha/(\sum^{n}_{j=1}v_{j}u_{j})$
      \State $\vec{x}\gets\vec{x}+t\vec{u}$
      \State $\vec{r}\gets\vec{r}-t\vec{u}$
      \State $\vec{w}\gets C^{-1}\vec{r}$
      \State $\beta\gets\sum^{n}_{j=1}w_{j}^{2}$
      \If{$\beta<\varepsilon_{\text{tol}}$}
        \If{$\|\vec{r}\|<\varepsilon_{\text{tol}}$}
          \State\Return $\vec{x}$
        \EndIf
      \EndIf
      \State $s\gets\beta/\alpha$
      \State $\vec{v}\gets C^{-1}\vec{w}+s\vec{v}$
      \State $\alpha\gets\beta$
    \EndFor
    \State\Fail ``Maximum number of iterations exceeded''
  \EndFunction
\end{algorithmic}
\end{algorithm}

Note that when we are working with a huge system of equations, the
preconditioning matrix $C$ is approximately equal to $L$ in the Cholesky
factorization $L\transpose{L}=A$. Then
$\transpose{(C^{-1})}C^{-1}\approx A^{-1}$ (up to errors due to
floating-point arithmetic).

Also note that we would run into these situations (a large system of
equations, mostly sparse, but positive-definite) when trying to solve
boundary-value ordinary differential equations.

For more information about the conjugate gradient method,
Kelley~\cite{kelley1995} spends about the first third of his book
discussing it.

\begin{theorem}
The $\vec{x}^{(k)}$ approximation from the conjugate-gradient method
without preconditioning has its absolute error
$\vec{e}^{(k)}=\vec{x}^{(k)}-\vec{x}$ satisfy the bounds
\begin{equation}
\|\vec{e}^{(k)}\|_{A}\leq2\left(\frac{\sqrt{\condition(A)}-1}{\sqrt{\condition(A)}+1}\right)^{k}\|\vec{e}^{(0)}\|_{A}
\end{equation}
where $\|\vec{e}\|_{A}=\sqrt{\langle\vec{e},A\vec{e}\rangle}$ is the
energy norm associated with the positive-definite matrix $A$.
\end{theorem}

See, e.g., Hackbusch~\cite[\S10.2.3]{hackbusch2016iterative} for details.