\chapter{Iterative Algorithms Solving Systems of Linear Equations}

\section{Jacobi and Gauss-Seidel}

We ultimately want to construct a sequence of vectors
$\{\vec{x}^{(k)}\}_{k\in\NN}$ which converges to the solution of a
system of linear equations $A\vec{x}=\vec{b}$. The first step will be to
write down some recurrence relation
\begin{equation}
\vec{x}^{(k+1)}=T\vec{x}^{(k)}+\vec{c}.
\end{equation}
How to do this?

We could begin by writing
\begin{equation}
A = L + D + U
\end{equation}
where $L$ is strictly lower-triangular, $D$ is diagonal, and $U$ is
strictly upper-triangular. Then we would take our system of equations
\begin{equation}
(L+D+U)\vec{x}=\vec{b},
\end{equation}
subtract $(L+U)\vec{x}$ from both sides to obtain,
\begin{equation}
D\vec{x}=-(L+U)\vec{x}+\vec{b},
\end{equation}
and finally multiply on the left by $D^{-1}$
\begin{equation}
\vec{x}=-D^{-1}(L+U)\vec{x}+D^{-1}\vec{b}.
\end{equation}
We then use this as our recurrence relation
\begin{equation}\label{eq:iterative-linear:jacobi-matrix-form}
\vec{x}^{(k+1)}=-D^{-1}(L+U)\vec{x}^{(k)}+D^{-1}\vec{b}.
\end{equation}
The presence of an inverse matrix might appear discouraging, but we
should recall the inverse of a diagonal matrix is just the inverse of
its components (which is easily computable!). So, we're good. We
strategy obtained from Eq~\eqref{eq:iterative-linear:jacobi-matrix-form}
is called the \define{Jacobi method}, and its basic structure may be
outlined in Algorithm~\ref{alg:iterative-linear:jacobi-iterative}.

If we don't have any good grounds for picking the initial approximation
$\vec{x}^{(0)}$, then the standard is to pick $\vec{x}^{(0)}=\vec{c}=D^{-1}\vec{b}$.

\begin{algorithm}\label{alg:iterative-linear:jacobi-iterative}
  \caption{Jacobi iterative method}
  \begin{algorithmic}[1]
    \Require $A=(a_{ij})$ is an $n\times n$ matrix
    \Require $a_{ii}\neq0$ for each $i=1,\dots,n$
    \Require $\vec{b}$ is an $n$ vector
    \Ensure $\vec{x}$ is a solution
    \Function{Jacobi}{$A$, $\vec{b}$, $\varepsilon_{\text{tol}}$, $N_{\text{max}}$}
    \State $\vec{p}\gets\vec{0}$ \algorithmiccomment{for previous iteration's guess}
    \For{$k=1$, \dots, $N_{\text{max}}$}
      \For{$i=1$, \dots, $n$}\label{alg:iterative-linear:jacobi-iterative:parallelizable-step}
        \State{$x_{i}\gets (b_{i} - \sum^{n}_{j\neq i}a_{ij}p_{j})/a_{ii}$}
      \EndFor
      \If{$\|\vec{x}-\vec{p}\|<\varepsilon_{\text{tol}}$}
        \State\Return $\vec{x}$
      \EndIf
      \State $\vec{p}\gets\vec{x}$
    \EndFor
    \State\Return $\vec{x}$
  \EndFunction
\end{algorithmic}
\end{algorithm}

\begin{chunk}
  The \FORTRAN/ implementation of Jacobi's method follows the algorithm
  we sketched out. An improved
  implementation would also pivot as needed to avoid any divide-by-zero
  problems. The
  \verb#jacobi_iterate# subroutine will store the approximation in the
  \verb#ans# parameter.\marginpar{\texttt{jacobi.f90}}
  \lstinputlisting[language=fortran18,firstline=27,lastline=57]{src/jacobi.f90}
\end{chunk}

One advantage of the Jacobi method is that we can parallelize the
inner-loop in
Step~\ref{alg:iterative-linear:jacobi-iterative:parallelizable-step}. The
problem is that we typically need many more steps for the Jacobi method
than alternatives.

The astute reader may wonder: why don't we just use the new values of
$x_{i}$ as we compute them? That is, why not compute
\begin{equation}
x_{i}=(b_{i}-\sum^{i-1}_{j=1}a_{ij}x_{j}-\sum^{n}_{j=i+1}a_{ij}p_{j})/a_{ii},
\end{equation}
instead of reusing the previous iteration's approximation throughout?
This is precisely the \define{Gauss-Seidel method}. Its skeleton may be
found in Algorithm~\ref{alg:iterative-linear:gauss-seidel}.

\begin{algorithm}\label{alg:iterative-linear:gauss-seidel}
  \caption{Gauss-Seidel iterative method}
  \begin{algorithmic}[1]
    \Require $A=(a_{ij})$ is an $n\times n$ matrix
    \Require $a_{ii}\neq0$ for each $i=1,\dots,n$
    \Require $\vec{b}$ is an $n$ vector
    \Ensure $\vec{x}$ is a solution
    \Function{Gauss-Seidel}{$A$, $\vec{b}$, $\varepsilon_{\text{tol}}$, $N_{\text{max}}$}
    \State $\vec{p}\gets\vec{0}$ \algorithmiccomment{for previous iteration's guess}
    \For{$k=1$, \dots, $N_{\text{max}}$}
      \For{$i=1$,\dots, $n$}
        \State{$x_{i}\gets (b_{i} - \sum^{i-1}_{j=1}a_{ij}x_{j}- \sum^{n}_{j=i+1}a_{ij}p_{j})/a_{ii}$}
      \EndFor
      \If{$\|\vec{x}-\vec{p}\|<\varepsilon_{\text{tol}}$}
        \State\Return $\vec{x}$
      \EndIf
      \State $\vec{p}\gets\vec{x}$
    \EndFor
    \State\Return $\vec{x}$
  \EndFunction
\end{algorithmic}
\end{algorithm}

\begin{chunk}
  The \FORTRAN/ implementation of Gauss-Seidel is straightforward. The
  only subtle bit is that at the start of each iteration, we don't need
  to initialize
  $\vec{x}^{(k+1)}=\vec{x}^{(k)}$ since it's already storing the
  previous iteration's results. So we can update the values
  in-place. Special attention must be paid that we are subtracting from
  $b_{i}$ the values $a_{ij}x_{j}$ for all $j\neq i$, and that this is
  stored in $x_{i}$. We still need \verb#prev_iter# for the stopping
  condition, which is why we keep it around.
  The same caveat about pivoting mentioned regarding
  the Jacobi implementation applies here as well. The
  \verb#gauss_seidel# subroutine will store the approximation in the
  \verb#ans# parameter.\marginpar{\texttt{jacobi.f90}}
  \lstinputlisting[language=fortran18,firstline=59,lastline=90]{src/jacobi.f90}
\end{chunk}

We have some estimate on the error bounds using the following result.

\begin{defn}\index{Spectral Radius}
Let $T$ be an $n\times n$ matrix over the reals\footnote{We could extend
this to the complex numbers, or restrict it to the rationals, or any
other subfield of $\CC$ if you want. But floating point numbers are
supposed to approximate $\RR$, so we are really interested in that field.}.
We define the \define{Spectral Radius} of $T$ to be the non-negative
real number $\rho(T) := \max\spec(T)$ equal to the maximum of the
absolute value of the eigenvalues of $T$.
\end{defn}

\begin{lemma}
The solution to $\vec{x}=T\vec{x}+\vec{c}$ is $\vec{x}=(I-T)^{-1}\vec{c}$.
Moreover the sequence $\{\vec{x}^{(k)}\}_{k\in\NN}$ defined by
$\vec{x}^{(k+1)}=T\vec{x}^{(k)}+\vec{c}$ converges to the solution if
and only if $\rho(T)<1$.
\end{lemma}

\begin{proof}[Proof sketch]
$(\Longleftarrow)$ Assume $\rho(T)<1$. We want to prove
convergence. Observe that 
\begin{align*}
\vec{x}^{(k)}
&=T\vec{x}^{(k-1)}+\vec{c}\\
&=T(T\vec{x}^{(k-2)}+\vec{c})+\vec{c}=T^{2}\vec{x}^{(k-2)}+(I+T)\vec{c}\\
&=\dots=T^{k}\vec{x}^{(0)}+(I+T+\dots+T^{k-1})\vec{c}.
\end{align*}
Taking the $k\to\infty$ limit gives us
\begin{equation}
\lim_{k\to\infty}\vec{x}^{(k)}=(I-T)^{-1}\vec{c},
\end{equation}
since $T^{k}\to0$ as $k\to\infty$. This gives us convergence.

$(\Longrightarrow)$ Assume that $\vec{x}^{(k)}\to\vec{x}$ for any
initial value $\vec{x}^{(0)}$ and $\vec{x}=T\vec{x}+\vec{c}$. Now let us
have $\vec{y}$ be any arbitrary value, and let us have
$\vec{x}^{(0)}=\vec{x}-\vec{y}$ be the initial value for the sequence.
Observe
\begin{equation}
\begin{split}
\vec{x}-\vec{x}^{(k)}&=(T\vec{x}+\vec{c})-(T\vec{x}^{(k-1)}+\vec{c})\\
&=T(\vec{x}-\vec{x}^{(k-1)}),
\end{split}
\end{equation}
so
\begin{equation}
\vec{x}-\vec{x}^{(k)}=T(\vec{x}-\vec{x}^{(k-1)})=T^{2}(\vec{x}-\vec{x}^{(k-2)})
=\dots=T^{k}(\vec{x}-\vec{x}^{(0)})=T^{k}\vec{y}.
\end{equation}
In particular, since we assumed the sequence $\vec{x}^{(k)}$ converges,
this means
\begin{equation}
\lim_{k\to\infty}T^{k}\vec{y}=\vec{0},
\end{equation}
which implies $\rho(T)<1$.
\end{proof}

\begin{prop}
If $\|T\|<1$ for any induced matrix norm and $\vec{c}$ is any given
vector, then the sequence $\{\vec{x}^{(k)}\}_{k\in\NN}$ converges to the
solution $\vec{x}$ for any initial guess $\vec{x}^{(0)}$. Furthermore,
we have the following error bounds:
\begin{enumerate}
\item $\|\vec{x}-\vec{x}^{(k)}\|\leq\|T\|^{k}\|\vec{x}^{(0)}-\vec{x}\|$;
\item $\displaystyle\|\vec{x}-\vec{x}^{(k)}\|\leq\frac{\|T\|^{k}}{1-\|T\|}\|\vec{x}^{(1)}-\vec{x}^{(0)}\|$.
\end{enumerate}
\end{prop}

\begin{proof}
Observe that
\begin{subequations}
\begin{equation}
\vec{x}=T\vec{x}+\vec{c}
\end{equation}
and
\begin{equation}
\vec{x}^{(k)}=T\vec{x}^{(k-1)}+\vec{c}.
\end{equation}
\end{subequations}
Therefore
\begin{equation}
\vec{x}-\vec{x}^{(k)}=(T\vec{x}+\vec{c})-(T\vec{x}^{(k-1)}+\vec{c})=T(\vec{x}-\vec{x}^{(k-1)}).
\end{equation}
Therefore
\begin{equation}
\vec{x}-\vec{x}^{(k)}=T^{k}(\vec{x}-\vec{x}^{(0)}).
\end{equation}
Taking the norm on both sides produces the first error bound.

Observe that
\begin{equation}
\begin{split}
\vec{x}^{(k+1)}-\vec{x}^{(k)}
&=(T\vec{x}^{(k)}+\vec{c})-(T\vec{x}^{(k-1)}+\vec{c})\\
&=T(\vec{x}^{(k)}-\vec{x}^{(k-1)}).
\end{split}
\end{equation}
Then by induction, we would have
\begin{equation}
\vec{x}^{(k+1)}-\vec{x}^{(k)}=T^{k}(\vec{x}^{(1)}-\vec{x}^{(0)}).
\end{equation}
So far, so good. Now, for any nonzero $m\in\NN$, $m\neq0$, we would have
\begin{align*}
\|\vec{x}^{(k+m)}-\vec{x}^{(k)}\|&=
\|\vec{x}^{(k+m)}-\vec{x}^{(k+m-1)}+\vec{x}^{(k+m-1)}-\dots+\vec{x}^{(k+1)}-\vec{x}^{(k)}\|\\
&\leq\|\vec{x}^{(k+m)}-\vec{x}^{(k+m-1)}\|+\|\vec{x}^{(k+m-1)}-\vec{x}^{(k+m-2)}\|+\dots+\|\vec{x}^{(k+1)}-\vec{x}^{(k)}\|\\
&\leq \|T^{k+m-1}\|\cdot\|\vec{x}^{(1)}-\vec{x}^{(0)}\|+\|T^{k+m-2}\|\cdot\|\vec{x}^{(1)}-\vec{x}^{(0)}\|+\dots+\|T^{k}\|\cdot\|\vec{x}^{(1)}-\vec{x}^{(0)}\|.
\end{align*}
Taking the $m\to\infty$ limit yields the result
\begin{equation}
  \begin{split}
\|\vec{x}-\vec{x}^{(k)}\|&\leq\|T^{k}\sum^{\infty}_{n=0}T^{n}\|\cdot\|\vec{x}^{(1)}-\vec{x}^{(0)}\|\\
&\leq\|T^{k}\|\left(\sum^{\infty}_{n=0}\|T^{n}\|\right)\|\vec{x}^{(1)}-\vec{x}^{(0)}\|=\frac{\|T^{k}\|}{1-\|T\|}\|\vec{x}^{(1)}-\vec{x}^{(0)}\|.
  \end{split}
\end{equation}
Note that submultiplicativity of the matrix norm is key in obtaining the
second line of the previous equation. Thus we obtain the second error bound.
\end{proof}

\section{Successive Over-Relaxation}

Now, Gauss-Seidel gives us the iterative method:
\begin{equation}
a_{ii}x_{i}^{(k)}=\underbrace{b_{i}-\sum^{i-1}_{j=1}a_{ij}x^{(k)}_{j}-\sum^{n}_{j=i}a_{ij}x_{j}^{(k-1)}}_{\mbox{looks like a residual}}
+a_{ii}x_{i}^{(k-1)}.
\end{equation}
We shall introduce some notation to assist our discussion.

Let us denote by $\vec{x}_{i}^{(k)}$ the $i^{\text{th}}$ step in
computing the $k^{\text{th}}$ iteratative approximation in Gauss-Seidel,
i.e.,
\begin{equation}
\vec{x}_{i}^{(k)}=(x_{1}^{(k)}, x_{2}^{(k)},\dots, x_{i-1}^{(k)},
x_{i}^{(k-1)}, \dots, x_{n}^{(k-1)}).
\end{equation}
Let us denote the residual for this vector be denoted by
\begin{equation}
\vec{r}_{i}^{(k)}=\vec{b}-A\vec{x}_{i}^{(k)}.
\end{equation}

Now, using this notation, we can write Gauss-Seidel as
\begin{equation}
a_{ii}x^{(k)}_{i} = a_{ii}x^{(k-1)}_{i} + r_{ii}^{(k)},
\end{equation}
equivalently
\begin{equation}
x^{(k)}_{i} = x^{(k-1)}_{i} + \frac{r_{ii}^{(k)}}{a_{ii}}.
\end{equation}
So far, we have just described Gauss-Seidel with different notation.

Successive over-relaxation introduces a new parameter $\omega>0$ and
modified Gauss-Seidel by
\begin{equation}
x^{(k)}_{i} = x^{(k-1)}_{i} + \omega\frac{r_{ii}^{(k)}}{a_{ii}}.
\end{equation}
When $\omega=1$ we recover Gauss-Seidel, when $\omega<1$ we have
under-relaxation (useful for obtaining convergence for a divergent
situation), and when $\omega>1$ we have over-relaxation. The problem
before us now is to pick an ``optimal'' $\omega$.

\begin{algorithm}\label{alg:iterative-linear:sor:sor}
  \caption{Successive over-relaxation}
  \begin{algorithmic}[1]
    \Require $A=(a_{ij})$ is an $n\times n$ matrix
    \Require $a_{ii}\neq0$ for each $i=1,\dots,n$
    \Require $0<\omega$
    \Require $\vec{b}$ is an $n$ vector
    \Ensure $\vec{x}$ is a solution
    \Function{Successive Over-Relaxation}{$A$, $\vec{b}$, $\vec{x}_{\text{init}}$, $\omega$, $\varepsilon_{\text{tol}}$, $N_{\text{max}}$}
    \State $\vec{p}\gets\vec{x}_{\text{init}}$ \algorithmiccomment{for previous iteration's guess}
    \For{$k=1$, \dots, $N_{\text{max}}$}
      \For{$i=1$, \dots, $n$}
        \State{$x_{i}\gets (1-\omega)p_{i} + \omega (b_{i} - \sum^{i-1}_{j=1}a_{ij}x_{j}- \sum^{n}_{j=i+1}a_{ij}p_{j})/a_{ii}$}
      \EndFor
      \If{$\|\vec{x}-\vec{p}\|<\varepsilon_{\text{tol}}$}
        \State\Return $\vec{x}$
      \EndIf
      \State $\vec{p}\gets\vec{x}$
    \EndFor
    \State\Return $\vec{x}$
  \EndFunction
\end{algorithmic}
\end{algorithm}

Before we analyze the convergence properties of successive
over-relaxation, let us review a few definitions from linear algebra.

\begin{definition}
We call an $n\times n$ matrix $A$ \define{Symmetric} if $\transpose{A}=A$,
and \define{Antisymmetric} (or \emph{skew-symmetric}) if $\transpose{A}=-A$.
\end{definition}

\begin{prop}
Let $A$ be an antisymmetric $n\times n$ matrix. Then for any
$\vec{x}\in\RR^{n}$ we have $\transpose{\vec{x}}A\vec{x}=0$.
\end{prop}

\begin{prop}
Let $M$ be any $n\times n$ real matrix. We can write $M$ uniquely as the
sum of a symmetric matrix $S$ and antisymmetric matrix $A$, i.e.,
$M=A+S$ where $A=(M-\transpose{M})/2$ and $S=(M+\transpose{M})/2$.
\end{prop}

\begin{definition}
Let $A$ be a symmetric $n\times n$ real matrix. We call $A$
\define{Positive-Definite} if for all (nonzero)
$\vec{x}\in\RR^{n}\setminus\{\vec{0}\}$ we have
$\transpose{\vec{x}}A\vec{x}>0$.
\end{definition}

\begin{rmk}
Note that, by the previous propositions in this section, we could
ostensibly weaken the notion of a positive-definite matrix to any
nonsingular matrix $M$ since
\begin{equation}
  \begin{split}
\transpose{\vec{x}}M\vec{x}&=\transpose{\vec{x}}S\vec{x}+\transpose{\vec{x}}A\vec{x}\\
&=\transpose{\vec{x}}S\vec{x},
  \end{split}
\end{equation}
where $A=(M-\transpose{M})/2$ and $S=(M+\transpose{M})/2$.
In practice, however, we must be careful because floating-point arithmetic
doesn't necessarily respect $\transpose{\vec{x}}A\vec{x}=0$.
\end{rmk}

\begin{theorem}[Kahan]
If $a_{ii}\neq0$ for each $i=1,\dots,n$, then
$\rho(T_{\omega})\geq|\omega-1|$ where if $A=D+L+U$ we have
\[ T_{\omega} = (D+\omega L)^{-1}[(1-\omega)D-\omega U]. \]
\end{theorem}

This was part of Kahan's 1958 doctoral thesis.

\begin{theorem}[{Ostrowski--Reich~\cite{ostrowski1954linear,reich1949convergence}}]
If $A$ is a [symmetric] positive-definite matrix and $0<\omega<2$, then
successive over-relaxation converges for any choice of initial $\vec{x}^{(0)}$.
\end{theorem}

\begin{theorem}
Let $T_{g}$ and $T_{j}$ be the iterative matrices associated with the
Gauss-Seidel and Jacobi methods, respectively. If $A$ is symmetric
positive-definite, then $\rho(T_{g})=[\rho(T_{j})]^{2}<1$ and the
optimal choice of $\omega$ for successive over-relaxation is
\begin{equation}\label{eq:iterative:sor:optimal-w-for-positive-definite-mat}
\omega = \frac{2}{1 + \sqrt{1 - [\rho(T_{j})]^{2}}}.
\end{equation}
With this choice, $\rho(T_{\omega})=\omega-1$.
\end{theorem}

\begin{rmk}
When $A=D+L+U$ satisfies $\det(\lambda D+zL+z^{-1}U)=\det(\lambda D+L+U)$
for any $z\in\CC\setminus\{0\}$ and $\lambda\in\CC$, then Eq~\eqref{eq:iterative:sor:optimal-w-for-positive-definite-mat}
gives the optimal $\omega$. This is the more general situation, since it
automatically describes tridiagonal matrices. The more general situation
was proven in Young's doctoral thesis~\cite{young1950} and may be found
in \S10.1 of Greenbaum~\cite{greenbaum1997iterative} as well as \S4.6.2 of Hackbusch~\cite{hackbusch2016iterative}.
\end{rmk}


\section{Conjugate Gradient Method}

\begin{definition}
We call an $n\times n$ matrix $A$ \define{Symmetric} if $\transpose{A}=A$,
and \define{Antisymmetric} (or \emph{skew-symmetric}) if $\transpose{A}=-A$.
\end{definition}

\begin{prop}
Let $A$ be an antisymmetric $n\times n$ matrix. Then for any
$\vec{x}\in\RR^{n}$ we have $\transpose{\vec{x}}A\vec{x}=0$.
\end{prop}

\begin{prop}
Let $M$ be any $n\times n$ real matrix. We can write $M$ uniquely as the
sum of a symmetric matrix $S$ and antisymmetric matrix $A$, i.e.,
$M=A+S$ where $A=(M-\transpose{M})/2$ and $S=(M+\transpose{M})/2$.
\end{prop}

\begin{definition}
Let $A$ be a symmetric $n\times n$ real matrix. We call $A$
\define{Positive-Definite} if for all $\vec{x}\in\RR^{n}$ we have $\transpose{\vec{x}}A\vec{x}>0$.
\end{definition}

\begin{rmk}
Note that, by the previous propositions in this section, we could
ostensibly weaken the notion of a positive-definite matrix to any
nonsingular matrix $M$ since
\begin{equation}
  \begin{split}
\transpose{\vec{x}}M\vec{x}&=\transpose{\vec{x}}S\vec{x}+\transpose{\vec{x}}A\vec{x}\\
&=\transpose{\vec{x}}S\vec{x},
  \end{split}
\end{equation}
where $A=(M-\transpose{M})/2$ and $S=(M+\transpose{M})/2$.
In practice, however, we must be careful because floating-point arithmetic
doesn't necessarily respect $\transpose{\vec{x}}A\vec{x}=0$.
\end{rmk}


