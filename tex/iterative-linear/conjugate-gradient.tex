\section{Conjugate Gradient Method}

\begin{definition}
We call an $n\times n$ matrix $A$ \define{Symmetric} if $\transpose{A}=A$,
and \define{Antisymmetric} (or \emph{skew-symmetric}) if $\transpose{A}=-A$.
\end{definition}

\begin{prop}
Let $A$ be an antisymmetric $n\times n$ matrix. Then for any
$\vec{x}\in\RR^{n}$ we have $\transpose{\vec{x}}A\vec{x}=0$.
\end{prop}

\begin{prop}
Let $M$ be any $n\times n$ real matrix. We can write $M$ uniquely as the
sum of a symmetric matrix $S$ and antisymmetric matrix $A$, i.e.,
$M=A+S$ where $A=(M-\transpose{M})/2$ and $S=(M+\transpose{M})/2$.
\end{prop}

\begin{definition}
Let $A$ be a symmetric $n\times n$ real matrix. We call $A$
\define{Positive-Definite} if for all $\vec{x}\in\RR^{n}$ we have $\transpose{\vec{x}}A\vec{x}>0$.
\end{definition}

\begin{rmk}
Note that, by the previous propositions in this section, we could
ostensibly weaken the notion of a positive-definite matrix to any
nonsingular matrix $M$ since
\begin{equation}
  \begin{split}
\transpose{\vec{x}}M\vec{x}&=\transpose{\vec{x}}S\vec{x}+\transpose{\vec{x}}A\vec{x}\\
&=\transpose{\vec{x}}S\vec{x},
  \end{split}
\end{equation}
where $A=(M-\transpose{M})/2$ and $S=(M+\transpose{M})/2$.
In practice, however, we must be careful because floating-point arithmetic
doesn't necessarily respect $\transpose{\vec{x}}A\vec{x}=0$.
\end{rmk}


