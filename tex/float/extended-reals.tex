\subsection{*Extended Real Number System}\label{subsec:computer:extended-real-number-system}

We should bear in mind that \ieee-754 floating-point arithmetic strives
to emulate the \emph{extended reals}
\begin{equation}
  \extendedRR = \RR\cup\{-\infty,+\infty\}.
\end{equation}
The usual intuition holds with arithmetic operators involving
infinities, except the following are ``indeterminate forms'' and illegal
(left undefined):
\begin{itemize}
\item $\infty-\infty$
\item $0\times(\pm\infty)$
\item $\pm\infty/\infty$ (and $\pm\infty/(-\infty)$).
\end{itemize}
Though it is useful in probability and measure theory to define
$0\times(\pm\infty)=0$.

\begin{axiom}[Addition]
  For any finite $x\in\RR$, we have
  $$x+\infty=\infty+x=\infty.$$
  Further
  $$\infty+\infty=\infty.$$
\end{axiom}


\begin{xca}
  Prove or find a counter-example: addition is commutative in the
  extended real-number system.
\end{xca}


\begin{axiom}[Subtraction]
  For any finite $x\in\RR$, we have
  $$x-\infty=-\infty+x=-\infty.$$
  Further, we have
  $$-\infty-\infty=-\infty.$$
\end{axiom}


\begin{axiom}[Multiplication]
  For any positive $x\in\RR$, $x>0$, we have
  $$x\times(\pm\infty)=(\pm\infty)\times x = \pm\infty$$
  and
  $$(-x)\times(\pm\infty)=(\pm\infty)\times(-x)=\mp\infty.$$
  This permits us to treat $-\infty$ as $-1\times(+\infty)$.
\end{axiom}


\begin{xca}
  Prove or find a counter-example: multiplication is commutative in the
  extended real-number system.
\end{xca}


\begin{axiom}[Division]
  If $x\in\RR$, then
  $$\frac{x}{\infty}=\frac{x}{-\infty}=0$$.
\end{axiom}

\begin{rmk}[Extended Reals with Dedekind Cuts]\index{Dedekind Cuts!Extended}
For real analysts (and lovers of set theoretic constructs), we could
define $\extendedRR$ using Dedekind cuts. Recall, a Dedekind cut is a
pair $(L,U)$ of subsets $L\subset\QQ$ and $U\subset\QQ$ such that
\begin{itemize}
\item $L\neq\emptyset$ and $U\neq\emptyset$
\item if $a\in L$ and $b\in U$, then $a<b$
\item if $a\in L$, then there is an $\ell\in L$ such that $a < \ell$
\item if $b\in U$, then there is some $u\in U$ such that $u < b$
\item if $a\in L$ and $b\in\QQ$ such that $b < a$, then $b\in L$
\item if $b\in U$ and $u\in\QQ$ such that $b < u$, then $u\in U$
\end{itemize}
Then a real number is defined as a cut $x := (L,U)$ such that every
$\ell\in L$ satisfies $\ell < x$, and every $u\in U$ satisfies $x < u$.

We can construct the extended reals $\extendedRR$ using ``extended
Dedekind cuts'', permitting (1) $L=\emptyset$ and $U=\QQ$, and (2)
$L=\QQ$ and $U=\emptyset$. These model $-\infty$ and $+\infty$,
respectively. All other conditions on Dedekind cuts (besides
non-emptiness) are imposed on extended Dedekind cuts, and we get a
rigorous construction of the extended real number system.
\end{rmk}


\begin{xca}
  Prove or find a counter-example: $\extendedRR$ is an ordered field, an
  ordered ring, or a group (using either multiplication or addition, whichever).
\end{xca}


\begin{xca}
  If we consider $\extendedRR^{*}=\extendedRR\cup\{u\}$ where $u$ is the
  special ``undefined'' constant, where we now have $\infty-\infty=u$,
  $0\times(\pm\infty)=u$, $(\pm\infty)/\infty=(\pm\infty)/(-\infty)=u$,
  and for any $a\in\extendedRR^{*}$ and any binary operation
  $\star\in\{+,-,\times,/\}$ we have $u\star a = a\star u = u$;
  then does $\extendedRR^{*}$ form a field? A ring? A group?
\end{xca}


\begin{xca}
  When working in $\extendedRR$, what should $\tanh(\pm\infty)$ be?
  [Hint: consider what $\exp(\pm\infty)$ should be, then use the
    definition of hyperbolic tangent.]
\end{xca}


\begin{xca}
  In the extended reals, what should $\arctan(\pm\infty)$ be?
\end{xca}
