\subsubsection{Identities of Floating-Point Arithmetic}

From the algorithms we have thus described, we have a number of useful
identities. This discussion stems from reading Knuth~\cite{taocp2}.

\begin{defn}
  We define a strict order relation $<$ on normal and subnormal
  floating-point numbers $u$ and $v$ by demanding $u < v$ if the
  corresponding rational numbers are thus ordered 
  $\fl^{-1}(u)<\fl^{-1}(v)$.

  Similarly, we define a strict equality $=$ on normal and subnormal
  floating-point numbers by demanding the corresponding rational numbers
  are equal. (Together, these define a partial order $\leq$ on normal
  and subnormal floating-point numbers.)
\end{defn}

\begin{lemma}[$\fl(-)$ preserves sign]\label{lemma:fl-preserves-sign}
  Let $a\in\RR$ be representable by a floating-point number. Then $a\geq 0$ 
  implies $\fl(a)\geq 0$, and $a\leq 0$ implies $\fl(a)\leq 0$.
\end{lemma}
\begin{proof}
  We have $u=\fl(a)$, then necessarily $\fl^{-1}(u) = a(1 + \varepsilon)$
  where $|\varepsilon| < \machEps$. So the sign of $u$ is the same as
  $a$, since $1 + \varepsilon > 1 - \machEps > 0$.
\end{proof}

\begin{lemma}[$\fl(-)$ preserves partial ordering]
  If $a,b\in\RR$ are representable by floating-point numbers, then
  $a\leq b$ implies $\fl(a)\leq\fl(b)$.
\end{lemma}
\begin{proof}
  If $a\leq 0\leq b$, then the result is immediate by Lemma \ref{lemma:fl-preserves-sign}.

  Without loss of generality, we suppose $0\leq a\leq b$. (The other
  case would be $a\leq b\leq 0$.)

  Let $u=\fl(a)$ and $v=\fl(b)$.

  We see that $u\geq0$ and $v\geq 0$. We also see that $w=\fl(b-a)\geq0$
  by Lemma \ref{lemma:fl-preserves-sign} and hypothesis $b\geq a$.
  The claim is that $v\ominus u=w$ by the fundamental axiom of
  floating-point arithmetic. Hence $v\ominus u\geq0$ implies $v\geq u$,
  or equivalently $\fl(b)\geq\fl(a)$.
\end{proof}
\begin{thm}
  We have the following identities for any floating-point numbers
  $u,v\in\mathbb{F}$:
  \begin{equation*}
    u\oplus v = v\oplus u.
  \end{equation*}
\end{thm}
\begin{proof}
  We see from \hyperref[alg:float-addition]{Algorithm A} that the first
  step reorders the arguments so the left summand is always greater than
  the right summand. This would mean the order of summands supplied to
  floating-point addition is irrelevant.
\end{proof}

\begin{thm}
  We have the following identities for any floating-point numbers
  $u,v\in\mathbb{F}$:
  \begin{equation*}
    u\ominus v = u \oplus -v.
  \end{equation*}
\end{thm}
\begin{proof}
This was how we defined floating-point subtraction in
\hyperref[alg:float-subtraction]{Algorithm S}.
\end{proof}

\begin{thm}
  We have the following identities for any floating-point numbers
  $u,v\in\mathbb{F}$:
  \begin{subequations}
    \begin{gather}
      %% u\oplus v = v\oplus u;\\
      %% u\ominus v = u \oplus -v;\\
      -(u\oplus v) = -u \oplus -v;\\
      u\oplus v=0\quad\mbox{if and only if}\quad v=-u;\\
      u\oplus 0=u.
    \end{gather}
    If further $u\oplus v=u+v$ is exact, then
    \begin{equation}
      (u\oplus v)\ominus v=u.
    \end{equation}
  \end{subequations}
\end{thm}
\begin{proof}
  The trick is to consider $x=\fl^{-1}(u)$ and $y=\fl^{-1}(v)$ the rational
  numbers corresponding to the given floating-point encodings. Then the
  claims boil down to invoking the fundamental axiom of floating-point:
  $-(u\oplus y)=\fl(-(x+x))=\fl((-x)+(-x))=(-u)\oplus(-v)$.

  Similarly, $u\oplus v=\fl(x+y)=0$ implies $x=\epsilon-y$ for any
  $|\epsilon|<\beta^{e_{\text{min}}-t}$ which is encoded as $\fl(x)=-\fl(y)$,
  as desired.
  
  And $u\oplus0=u$ follows from $\fl(x+\epsilon)=\fl(x)$ for any $|\epsilon|<\machEps|x|$.

  The claim $(u\oplus v)\ominus v=u$ can be encoded as
  $\fl(\fl^{-1}(u\oplus v)-v)$ but since $u\oplus v$ is exact,
  $\fl^{-1}(u\oplus v) = x + y$. Hence we have $\fl((x+y)-y)=\fl(x)$
  identically. 
\end{proof}

\begin{thm}
  If $u\leq v$ are arbitrary floating-point numbers, then for any
  floating-point number $w$ we have
  \begin{equation}
    u\oplus w\leq v\oplus w.
  \end{equation}
\end{thm}
\begin{proof}
  The trick is to realize that there are real numbers $x$ and $y$ such
  that $\fl(x)\leq\fl(y)$, then $\fl(x)\oplus\fl(z)\leq\fl(y)\oplus\fl(z)$
  for any representable real $z$. We recall $\fl(x)\oplus\fl(z) = \fl(x+z)$
  and $\fl(y)\oplus\fl(z)=\fl(y+z)$ by the fundamental axiom of
  floating-point. Hence $\fl(x+z)\leq \fl(y+z)$.
\end{proof}

\begin{xca}[{Knuth~\cite[(4.2.2, ex 1)]{taocp2}}]
  Prove for any floating-point numbers $u$ and $v$ that
  $u\ominus v=-(v\ominus u)$.
\end{xca}

\begin{thm}[Floating-point multiplication is commutative]
  For any floating-point numbers $u$ and $v$, we have
  \begin{equation*}
    u\otimes v=v\otimes u.
  \end{equation*}
\end{thm}
\begin{proof}
  This follows from \hyperref[alg:float-multiplication]{Algorithm M}
  where we'd have to prove each possible case ($u$ normal, $v$ normal;
  $u$ a \NaN, $v$ a normal, etc.). The only interesting cases are when $u$
  and $v$ are normal or subnormal. We see from the algorithm that the
  components of the ``unnormalized'' product are commutative
  $\XOR(s_{x},s_{y})=\XOR(s_{y},s_{x})$, $m_x m_y = m_y m_x$,
  $e_x + e_y = e_y + e_x$. Hence the same ``unnormalized''
  floating-point number is fed to algorithm N, which produces the same
  results.
\end{proof}

\begin{thm}
  For any floats $u$ and $v$, we have
  \begin{equation*}
    (-u)\otimes v = u\otimes(-v) =  -(u\otimes v).
  \end{equation*}
\end{thm}
\begin{proof}
  Let $x=\fl^{-1}(u)$, $y=\fl^{-1}(v)$ be the rational numbers
  corresponding to the floating-point numbers. We see then that
  $\fl((-x)\times y) = \fl(x\times(-y)) = -\fl(x\times y)$.
\end{proof}

\begin{thm}
  For any floating-point number $u$, we have
  \begin{equation*}
    1\otimes u = u\otimes 1 = u.
  \end{equation*}
\end{thm}
\begin{proof}
  We have $1=\fl^{-1}(1)$ and $x=\fl^{-1}(u)$ be the rational numbers
  corresponding to the floating-point numbers given. Then we see
  $\fl(1\times x)=\fl(x\times 1)=\fl(x)$.
\end{proof}

\begin{thm}
  For any floats $u$ and $v$, we have
  \begin{equation*}
    u=0\mbox{ or }v=0\quad\mbox{implies}\quad u\otimes v=0.
  \end{equation*}
\end{thm}
\begin{proof}
  We see $\fl(0\times y)=\fl(0)=0$ and $\fl(x\times0)=\fl(0)=0$.
  Alternatively, look at step M3 in \hyperref[alg:float-multiplication]{Algorithm M}.
\end{proof}

\begin{rmk}
  The other direction does not hold in general. That is to say, we could
  have $u\otimes v=0$ but $u\neq0$ and $v\neq0$. How? Well, if
  $u=\beta^{e_{\text{min}}+1}$ and $v=\beta^{e_{\text{min}}+1}$, then $u\otimes v=\beta^{2e_{\text{min}}+2}<\beta^{e_{\text{min}}}$
  for $e_{\min}<-1$. Hence normalization would round $u\otimes v$ to zero.
  That is to say, when $u\otimes v$ underflows, we'd have the situation
  where $u\neq0$ and $v\neq0$ and $u\otimes v=0$.
\end{rmk}

\begin{thm}[Monotony law for multiplication]
  If $u\leq v$ are floating-point numbers and $w>0$ is a floating-point
  number, then $u\otimes w\leq v\otimes w$.
\end{thm}
\begin{proof}
  Let $x=\fl^{-1}(u)$ and $y=\fl^{-1}(v)$. Then $x\leq y$ by hypothesis.
  Let $z=\fl^{-1}(w)>0$.
  Then $xz\leq yz$, and preservation of partial ordering gives the result
  $\fl(x\times z)\leq \fl(y\times z)$.
\end{proof}

\begin{thm}
  For any floating-point numbers $u$ and $v$, we have
  \begin{equation*}
    (-u)\oslash v = u\oslash(-v)=-(u\oslash v).
  \end{equation*}
\end{thm}
\begin{proof}
  Let $x=\fl^{-1}(u)$ and $y=\fl^{-1}(v)$ be the rational numbers
  represented by the floats. Then we find
  $\fl((-x)/y)=\fl(x/(-y))=\fl(-(x/y)) = -(u\oslash v)$.
\end{proof}

\begin{thm}
  For any floating-point number $v$, we have
  \begin{equation*}
    0\oslash v = 0.
  \end{equation*}
\end{thm}
\begin{proof}
  Let $y=\fl^{-1}(v)$ be the rational number
  represented by the float. Then we find
  $0\oslash v = \fl(0/y) = \fl(0) = 0$.
\end{proof}

\begin{thm}
  For any floating-point number $u$, we have
  \begin{equation*}
    u\oslash 1 = u.
  \end{equation*}
\end{thm}
\begin{proof}
  Let $x=\fl^{-1}(u)$ be the rational number
  represented by the float. Then we find
  $u\oslash 1 = \fl(x/1) = \fl(x) = u$.
\end{proof}


\begin{thm}
  For any floating-point number $u$, we have
  \begin{equation*}
    u\oslash u = 1.
  \end{equation*}
\end{thm}
\begin{proof}
  Let $x=\fl^{-1}(u)$ be the rational number
  represented by the float. Then we find
  $u\oslash u = \fl(x/x) = \fl(1) = 1$.
\end{proof}

\begin{thm}
  For any floating-point numbers $u$ and $v$, if $u\otimes v=u\times
  v\neq0$ is exact and nonzero, then
  \begin{equation*}
      (u\otimes v)\oslash v=u.
  \end{equation*}
\end{thm}
\begin{proof}
  Let $x=\fl^{-1}(u)$ and $\fl^{-1}(v)=y$.
  We have $\fl(x\times y)=u\otimes v$ by hypothesis that $u\otimes v$ is
  exact.
  Let $z=x\times y = \fl^{-1}(u\otimes v)$.
  Then $\fl(z)\oslash v = \fl(z/y)$ by the fundamental axiom of
  floating-point arithmetic, and $\fl(z/y) = \fl((x\times y)/y)$ by
  hypothesis of exactness. Then we have $\fl((x\times y)/y) = \fl(x)$.
  Hence the result.
\end{proof}

\begin{thm}[Monotony laws of division]
  For any floating point numbers $u\leq v$ and $w>0$, then
  \begin{subequations}
    \begin{equation}
      u\oslash w\leq v\oslash w
    \end{equation}
    and further, when $0 < u \leq v$, we have
    \begin{equation}
      w\oslash u\geq w\oslash v
    \end{equation}
  \end{subequations}
\end{thm}
\begin{proof}
  Let $x=\fl^{-1}(u)$, $y=\fl^{-1}(v)$, $z=\fl^{-1}(w)>0$ be the
  rational numbers corresponding to the given floats. We have $x\leq y$
  by hypothesis.
  
  (1) We find that $x/z\leq y/z$, hence $\fl(x/z)\leq\fl(y/z)$.

  (2) When $0<x\leq y$, we have $z/x\geq z/y$. Hence $\fl(z/x)\geq\fl(z/y)$.
\end{proof}

\begin{prop}[Failure of Cauchy's law]\index{Cauchy--Schwarz inequality!Failure}
Let $x_{1},\dots,x_{n},y_{1},\dots,y_{n}$ be floating-point
numbers. Then, in general, Cauchy's law fails to hold:
\begin{equation}
  (x_{1}^{2}\oplus\dots\oplus x_{n}^{2})\otimes(y_{1}^{2}\oplus\dots\oplus y_{n}^{2})\not\geq
  (x_{1}\otimes y_{1})\oplus(\dots)\oplus(x_{n}\otimes y_{n}).
\end{equation}
Here $x_{j}^{2}=(x_{j}\otimes x_{j})$ and similarly for $y_{j}$, for $j=1,\dots,n$.
\end{prop}
\begin{proof}
  We just need to find a single counter-example.
  Take $n=2$, $x_{1}=x_{2}=1$, $y_{1}=1$, $y_{2} = 1 - \machEps/2$. Then
  $y_{1}\oplus y_{2}=2y_{1}$
  and $y_{1}^{2}\oplus y_{2}^{2} = 2\ominus\machEps$.
  Hence $(y_{1} \oplus y_{2})^{2} = 4$. Then Cauchy's inequality should
  give us $2(y_{1}^{2}\oplus y_{2}^{2})\geq(y_{1} + y_{2})^{2}$ but
  instead we find $2(2\ominus\machEps)\not\geq 4$. For single-precision
  arithmetic, we find we get $3.99999976\not\geq 4.0$, violating Cauchy.
\end{proof}


