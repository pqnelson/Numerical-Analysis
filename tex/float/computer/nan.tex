\subsubsection{Encoding NaN}

The \ieee-754 standard attempts to emulate \emph{real} numbers. There
are times when we could produce an indeterminate quantity (e.g., zero divided
by zero) or a complex quantity (e.g., $\sqrt{-1}$). The standard models
these quantities as \emph{not a number}, i.e., a special quantity
denoted \NaN.

We have two sorts of \NaN\ numbers: signaling and quiet. A signalling
\NaN\ raises an exception interrupting computation, a quiet one does
not. The \ieee-754 standard specifies \NaN\ is encoded as $(s,m,e_{\text{max}})$
for $m>0$ and $s\in\{0,1\}$ arbitrary, but does not specify how to
distinguish a quiet from a signaling \NaN. The 2008 revision suggests
\begin{enumerate}
\item For $\beta=2$, the most significant bit of the mantissa should be an
  ``\verb|is_quiet|'' flag (which is 1 for quiet \NaN\ and 0 for signaling)
\item For $\beta=10$, the top five bits of the combination field after
  the sign bit is set to 1. The sixth bit of the field is the ``\verb|is_quiet|''
  flag.
\end{enumerate}
We must stress that different processors have different conventions. For
example, Intel has the second most significant bit of the mantissa set
to 1 for ``quiet \NaN\ Floating-Point Indefinite''.

\begin{defn}
  A \define{Not-a-Number}\index{NaN@\NaN|see {Not-a-Number}} (or ``\emph{NaN}'') is a floating-point number
  $(s,m,e)$ such that $e=e_{\text{max}}$ and $m>0$.
\end{defn}

\begin{defn}
  Let $(s,m,e_{\text{max}})$ be an \NaN. Then we call $m$ the \define[Not-a-Number!Payload]{Payload}.
\end{defn}

Now we will list when \NaN\ will result from a computation. This is not
intended to be exhaustive!

\begin{axiom}
  The quotient of zero with zero, or infinity with infinity, is \NaN:
  \begin{equation}
    \frac{\pm0}{0} = \frac{\pm \infty}{\infty} = \NaN.
  \end{equation}
\end{axiom}

\begin{axiom}
  Multiplying infinity by zero produces a \NaN:
  \begin{equation}
    (\pm 0)\otimes(\infty)=(\pm0)\otimes(-\infty)=\NaN.
  \end{equation}
\end{axiom}

\begin{axiom}
  Adding infinities of different signs (or subtracting infinities of the
  same sign) produces \NaN:
  \begin{subequations}
    \begin{equation}
      \infty\oplus(-\infty)=\infty\ominus\infty=\NaN
    \end{equation}
    \begin{equation}
      (-\infty)\oplus\infty = \NaN.
    \end{equation}
  \end{subequations}
\end{axiom}

\begin{axiom}
If $u$ and $v$ are floating-point numbers and at least one of them is a \NaN,
then $u\oplus v$, $u\ominus v$, $u\otimes v$, $u\oslash v$ all produce \NaN.
\end{axiom}

\begin{axiom}
  In any computation where a complex number should be produced (e.g.,
  $\sqrt{-4}$ and $\sqrt{-\infty}$), a \NaN\ shall be produced.
\end{axiom}

\begin{defn}
  We denote the \define[Not-a-Number!Quiet]{Quiet NaN}\index{qNaN@\qNaN|see {Not-a-Number, Quiet}} as \qNaN.
\end{defn}

\begin{defn}
  We denote the \define[Not-a-Number!Signaling]{Signaling NaN}\index{sNaN@\sNaN|see {Not-a-Number, Signaling}} as \sNaN.
\end{defn}

\begin{defn}
  \ieee-754 defines the predicates for any floating-point number $x$,
  \begin{itemize}
  \item\index{$\isNaN(x)$} $\isNaN(x)$ return ``true'' if and only if $x=\NaN$.
  \item\index{$\isSignaling(x)$} $\isSignaling(x)$ return ``true'' if and only if $x=\sNaN$.
  \end{itemize}
\end{defn}

\begin{rmk}
  \FORTRAN/-1998 has the \verb|ieee_arithmetic| module\footnote{See,
  e.g., \url{http://fortranwiki.org/fortran/show/ieee_arithmetic}}
  contain \verb|ieee_is_nan(x)| for the $\isNaN(x)$
  predicate, and \CEE/ has \verb|isnan(arg)| defined in \verb|<math.h>|.
\end{rmk}
