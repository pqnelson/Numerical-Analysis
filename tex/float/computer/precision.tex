\subsubsection{Bounds on Precision and Accuracy}

\begin{defn}
  In the floating-point system $\mathbb{F}_{\beta,t,e_{\text{min}},e_{\text{max}}}$
  with rounding-scheme $R\colon\RR\to\mathbb{F}_{\beta,t,e_{\text{min}},e_{\text{max}}}$,
  we define the \define{Machine Precision}\index{$\machinePrec$} as the real number
  $\machinePrec\in\RR$
  which is the bound on rounding-error.
\end{defn}


\begin{ex}
For truncation $RZ(-)$, the machine precision is
\begin{equation}
  \machinePrecision = \beta^{1-t}
\end{equation}
by Lemma~\ref{lemma:idealized:truncation-round-off-error}.
\end{ex}


\begin{ex}
For round-to-nearest $RN$, the machine precision is
\begin{equation}
  \machinePrecision = \frac{\beta^{1-t}}{2}
\end{equation}
by Lemma~\ref{lemma:idealized:rn-round-off-error}. Since this is the
default rounding scheme for \ieee-754, this should be the quantity we
remember. We also observe this corresponds with our
Definition~\ref{defn:idealized:machine-epsilon} of machine epsilon,
carried over to \ieee-754 floating-point.
\end{ex}


\begin{defn}[{{\it Handbook\/}~\cite[(Def.2.6)]{10.5555/3235984}}]
  If $x\in\RR$ is representable by a finite floating-point number
  $u=\fl(x)$ such that $|u|\neq\Omega$ (so there is a next consecutive
  floating-point number and a preceding floating-point number),
  if $\beta^{n}\leq |u| < \beta^{n+1}$
  we define the \index{$\ulp(x)$}\define{Unit in Last Place} of $x$ to be the real-number
  $\ulp(x) = \beta^{n-(t-1)}$.
\end{defn}

\begin{rmk}
I should have some discussion about the definition of $\ulp(-)$, which
is another muddled concept. See, e.g., Muller~\cite{muller2005ulp} for a
discussion of the various definitions. Knuth~\cite[pp.232--233]{taocp2}
defines $\beta^{1-t}$ as an ulp, but this seems to be antiquated terminology.
Harrison~\cite{harrison-hol99} seems to be the first to formalize the
notion in a theorem prover.
\end{rmk}

\begin{defn}\index{$\ULP(u)$}
  If $u=\fl(x)$, we define $\ULP(u)=\ulp(x)$.
\end{defn}

\begin{ex}
  This $\ulp(-)$ function depends on the rounding scheme.
  We see that $\ulp(\RN(1)) = 2\machinePrec$ and $\ulp(\RZ(1))=\machinePrec$.
\end{ex}


\begin{thm}
  If $\fl(x)=(s,m,e)$, then $\ulp(x) = \beta^{\max(e,e_{\text{min}})-(t-1)}$.
\end{thm}


\begin{thm}
  In base $\beta=2$, we have for any $x\in\RR$ and float $u$,
  \begin{equation*}
    |\fl^{-1}(u)-x| < \frac{1}{2}\ulp(x)\implies u = \fl(x).
  \end{equation*}
\end{thm}


\begin{thm}
  For any real number $x\in\RR$ representable by a finite floating-point
  number $u:=\fl(x)$, we have
  \begin{equation*}
    |\fl^{-1}(u)-x| < \frac{1}{2}\ulp(x).
  \end{equation*}
\end{thm}


\begin{thm}
  For any real number $x\in\RR$ representable by a finite floating-point
  number $u:=\fl(x)$, we have
  \begin{equation*}
    |\fl^{-1}(u)-x| < \frac{1}{2}\ULP(u).
  \end{equation*}
\end{thm}


\begin{thm}
  For any real number $x\in\RR$ representable by a finite floating-point
  number $u=\RD(x)$ or $u=\RU(x)$, we have
  \begin{equation*}
    |\fl^{-1}(u)-x|\leq\ulp(x).
  \end{equation*}
\end{thm}


\begin{thm}
  For any real number $x\in\RR$ representable by a finite floating-point
  number $u=\RD(x)$ or $u=\RU(x)$, we have
  \begin{equation*}
    |\fl^{-1}(u)-x|\leq\ULP(u).
  \end{equation*}
\end{thm}
