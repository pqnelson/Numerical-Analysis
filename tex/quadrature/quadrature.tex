\section{Numerical Integration}

There are two ways to approximate the integral
\begin{equation}
I = \int^{b}_{a}f(x)\,\D x.
\end{equation}
We can either approximate the integrand, or we can approximate the
integration process.

The first way uses polynomial interpolation for $f(x)\approx p(x)$
relying on the methods we developed in Chapter~\ref{ch:polynomial-interpolation},
then we know how to integrate a polynomial. This gives us the ability to
estimate the error of this approximation, since polynomial interpolation
gives us $f(x)=p(x)+R$.

The second approach uses Riemann sums to approximate the integral, which
relies on knowing real analysis. While useful, it provides both
insight and a ``consistency check'' on results obtained via polynomial
interpolation. However, such an approach lies a bit out of scope for
these notes.

\subsection{Newton--Cotes Quadrature}

The first family of methods work with polynomial interpolation passing
through $x_{j}=x_{0}+j\,\Delta x$ where $\Delta x = (b-a)/n$. That is to
say, it passes through $n+1$ equally spaced points $x_{j}$.

\begin{example}
As a warm-up, let us consider the simplest interpolation: a linear
interpolation passing through $f(a)$ and $f(b)$. That is to say, we have
$x_{0}=a$, $x_{1}=b$, and $h=b-a$, with the interpolating polynomial:
\begin{subequations}
\begin{equation}
p(x) = \frac{(x-x_{1})}{(x_{0}-x_{1})}f(x_{0}) + \frac{(x-x_{0})}{(x_{1}-x_{0})}f(x_{1}).
\end{equation}
That is to say, when we include the explicit error term, we have
\begin{equation}
f(x)=p(x)+\frac{1}{2}f''(\xi(x))(x-x_{0})(x-x_{1}).
\end{equation}
Integrating both sides with respect to $x$ from $a$ to $b$,
\begin{equation}
\int^{b}_{a}f(x)\,\D x=\int^{b}_{a}p(x)\,\D x + \frac{f''(\xi)}{2}\int^{b}_{a}(x-x_{0})(x-x_{1})\,\D x,
\end{equation}
where we used the weighted mean-value theorem to pull $f''(\xi(x))$ out
in front of the error term's integral.
We can calculate the integral for the error term
\begin{calculation}
  \int^{b}_{a}(x-x_{0})(x-x_{1})\,\D x
\step{expand the polynomial}
  \int^{b}_{a}(x^{2}-(x_{0}+x_{1})x+x_{0}x_{1})\,\D x
\step{power rule for integral}
  \left.\frac{x^{3}}{3}-\frac{(x_{0}+x_{1})}{2}x^{2}+x_{0}x_{1}x\right|^{b}_{a}
\step{substitute the limits of integration, algebra}
  \frac{b^{3}-a^{3}}{3}-\frac{1}{2}(b^{3}-a^{3})-\frac{1}{2}(ab^{2}-a^{2}b)+(ab^{2}-a^{2}b)
\step{simplification}
  \frac{-(b^{3}-a^{3})}{6}+\frac{1}{2}(ab^{2}-a^{2}b)
\step{using binomial theorem}
  \frac{-(b-a)^{3}}{6}
\end{calculation}
This gives us
\begin{equation}
\int^{b}_{a}f(x)\,\D x=\int^{b}_{a}p(x)\,\D x - \frac{(b-a)^{3}f''(\xi)}{12}.
\end{equation}
Similarly, calculating the integral for the linear polynomial $p(x)$
gives us
\begin{calculation}
  \int^{b}_{a}p(x)\,\D x
\step{since $\int(x-c)\,\D x=(x-c)^{2}/2$}
  \left[\frac{(x-b)^{2}}{2(a-b)}f(a)+\frac{(x-a)^{2}}{2(b-a)}f(b)\right]^{b}_{a}
\step{substitute the limits of integration}
  \left[0-\frac{(a-b)^{2}}{2(a-b)}f(a)+\frac{(b-a)^{2}}{2(b-a)}f(b)-0\right]
\step{algebra}
  \left[-\frac{(a-b)}{2}f(a)+\frac{(b-a)}{2}f(b)\right]
\step{distributivity}
  \frac{b-a}{2}\left[f(b)-f(a)\right]
\end{calculation}
Therefore we obtain:
\begin{equation}
\int^{b}_{a}f(x)\,\D x=\frac{b-a}{2}\left[f(b)-f(a)\right] - \frac{(b-a)^{3}f''(\xi)}{12}.
\end{equation}
This is known as the \define{Trapezoid Rule}.
\end{subequations}
\end{example}

\begin{example}
Now, we can use the quadratic Lagrangian polynomial interpolation
passing through the points $x_{0}=a$, $x_{2}=b$, and the midpoint
$x_{1}=m=(b+a)/2$. We have
\begin{subequations}
\begin{multline}
f(x)=f(a)\frac{(x-m)(x-b)}{(a-m)(a-b)}+f(m)\frac{(x-a)(x-b)}{(m-a)(m-b)}
+f(b)\frac{(x-a)(x-m)}{(b-a)(b-m)}\\
+\frac{f'''(\xi(x))}{3!}(x-a)(x-m)(x-b).
\end{multline}
We see that the error term's polynomial factor $(x-a)(x-m)(x-b)$ is
positive on $[a,m]$ and negative on $[m,b]$. So to apply the weighted
mean-value theorem for the integral of the error term, we need to break
it up as
\begin{multline}
\int^{b}_{a}\frac{f'''(\xi(x))}{3!}(x-a)(x-m)(x-b)\,\D x\\
=\int^{m}_{a}\frac{f'''(\xi(x))}{3!}(x-a)(x-m)(x-b)\,\D x\\
+\int^{b}_{m}\frac{f'''(\xi(x))}{3!}(x-a)(x-m)(x-b)\,\D x.
\end{multline}
Direct calculation gives us
\begin{calculation}
  \int^{m}_{a}\frac{f'''(\xi(x))}{3!}(x-a)(x-m)(x-b)\,\D x
  +\int^{b}_{m}\frac{f'''(\xi(x))}{3!}(x-a)(x-m)(x-b)\,\D x
\step{weighted mean-value theorem}
  \frac{f'''(\xi_{1})}{3!}\int^{m}_{a}(x-a)(x-m)(x-b)\,\D x
  +\frac{f'''(\xi_{2})}{3!}\int^{b}_{m}(x-a)(x-m)(x-b)\,\D x
\step{performing the integrals}
  \frac{f'''(\xi_{1})}{3!}\frac{(a-b)^{4}}{64}
  +\frac{f'''(\xi_{2})}{3!}\frac{(-1)(a-b)^{4}}{64}
\step{distributivity}
  \bigl(f'''(\xi_{1})-f'''(\xi_{2})\bigr)\frac{(a-b)^{4}}{3!64}
\step{mean-value theorem}
  \bigl(\xi_{1}-\xi_{2}\bigr)f^{(4)}(\xi)\frac{(a-b)^{4}}{3!64}
\end{calculation}
Now, since $a\leq\xi_{1}\leq\xi_{2}\leq b$ we have
$0\leq\xi_{2}-\xi_{1}\leq\xi_{2}-a\leq b-a$, and then we can take the absolute
value of the error term to give us the bound:
\begin{equation}
\left|\int^{b}_{a}\frac{f'''(\xi(x))}{3!}(x-a)(x-m)(x-b)\right|\leq\left|\frac{h^{5}f^{(4)}(\xi)}{3!2}\right|,
\end{equation}
where $h=(b-a)/2$ and $h^{5}=(b-a)/32$.

Now, we can evaluate the integral for the
interpolating polynomial, first we observe
\begin{equation}
\int^{b}_{a}(x-c_{1})(x-c_{2})\,\D x
=\frac{b^{3}-a^{3}}{3}-\frac{1}{2}(b^{2}-a^{2})(c_{1}+c_{2})+(b-a)c_{1}c_{2}.
\end{equation}
Then we find
\begin{align}
\int^{b}_{a}f(a)\frac{(x-m)(x-b)}{(a-m)(a-b)}\D x &= f(a)\frac{b-a}{6}\\
\int^{b}_{a}f(m)\frac{(x-a)(x-b)}{(m-a)(m-b)}\D x &= f(m)\frac{2}{3}(b-a)\\
\int^{b}_{a}f(b)\frac{(x-a)(x-m)}{(b-a)(b-m)}\D x &= f(b)\frac{b-a}{6}.
\end{align}
\end{subequations}
Therefore, combining all these terms together, we get
\begin{subequations}
\begin{equation}
\int^{b}_{a}f(x)\,\D x = \frac{b-a}{6}\left[f(a)+4f(m)+f(b)\right] + R
\end{equation}
where the error term $R$ is bounded by
\begin{equation}
|R|\leq\left|\frac{h^{5}f^{(4)}(\xi)}{12}\right|,
\end{equation}
\end{subequations}
where $\xi\in[a,b]$ and $h=(b-a)/2$. This is \define{Simpson's Rule}.
A slightly tighter expression may be found for the error,
$R=-h^{5}f^{(4)}(\xi)/90$, but the important thing is not the
coefficient $1/90$ or $1/12$.
\end{example}

\begin{definition}
The \define{Degree of Precision} for a numerical integral method is the
greatest integer $n$ such that every polynomial $x^{k}$ is integrated
exactly for each $k=1,2,\dots,n$.
\end{definition}

We see that the trapezoid rule has its error directly proportional to
$f''(\xi)$, which means it is of degree of precision 1. On the other
hand, Simpson's method has error directly proportional to $f^{(4)}(\xi)$
which means any polynomial of degree less than 4 is integrated exactly
--- so Simpson's method has a degree of precision equal to 3.