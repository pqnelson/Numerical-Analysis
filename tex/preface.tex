\chapter{Preface}

This is a collated collection of my notes on numerical analysis.
The goal is to get the computer to numerically solve problems
encountered in ``continuous mathematics''.

I don't know if this replaces, e.g., Burden and Faires' book. My goal
was to have a coherent, self-contained explanation of the various
algorithms and code snippets.

\section*{Assumptions on the Reader}

I'm assuming the reader has familiarity with mathematical analysis (at
least: real analysis, complex analysis, and hopefully Fourier analysis)
and linear algebra. I suppose I could just assert the reader should read
Rudin, and return when finished, but then I'd never see you again!

I hope the reader has an understanding of real analysis at the level of
Abbott's \emph{Understanding Analysis}, and higher-dimensional real
analysis at the level of the first couple chapters of Munkres's
\emph{Analysis on Manifolds}.

If the reader has read through Axler's \emph{Linear Algebra Done Right},
that's wonderful and about the level I'm expecting.

\subsection*{Notation}

Further, the following notation should be familiar and understood:

\begin{itemize}
\item $f\colon A\into B$ indicates $f$ is an injection of $A$ into $B$;
\item $f\colon A\onto B$ indicates $f$ is a surjection of $A$ onto $B$;
\item $f\colon A\iso B$ is unorthodox notation indicating $f$ is an
  isomorphism or bijection of $A$ to $B$;
\item $A\subset B$ indicates $A$ is a subset of $B$, possibly $A=B$;
  and $B\supset A$ is a synonym for $A\subset B$;
\item $A\propersubset B$ indicates $A$ is a proper subset of $B$ (i.e.,
  $A\subset B$ and $A\neq B$), and $B\propersupset A$ is a synonym for
  $A\propersubset B$;
\item $\NN\propersubset\NN_{0}\propersubset\ZZ\propersubset\QQ\propersubset\RR\propersubset\CC$
  relates the number systems, where $\NN = \{1,2,\dots\}$ and $\NN_{0}=\{0\}\cup\NN$;
\item $\E\approx 2.71\dots$ is the base of the natural logarithm;
\item $\I=\sqrt{-1}$.
\end{itemize}

\section*{Programming Language \textit{de Rigueur}}

The programming language of choice for numerical analysis remains
\FORTRAN/, despite what Julia fanatics claim. Consequently, I write
modern \FORTRAN/-90 code implementing the algorithms discussed. This is
done using literate programming (thanks to Norman Ramsey's {\tt noweb}
toolkit).

I'm leaning on Knuth's literate programs written in WEB as an example,
since \FORTRAN/-90 greatly resembles \PASCAL/ in many ways. Having a
lack of research on ``What makes a program understandable'' complicates
things, so I'm relying on examples from authority.

\section*{Mathematical Registers}

I'm defaulting to the typical mathematical style where the text is
organized into ``chunks of text'' with a label prefixing the chunk. The
sorts of labels the reader may expect to find include classic favorites
(``definition'', ``theorem'', ``proof'') as well as ``example'',
``exercise'', ``lemma'', ``proposition'', ``corollary'', and many new
ones, too.

We try to be explicit with the introduction of new notation by using a
``notation'' register. If we stick to a convention (e.g., working with
double-precision by default), we'll announce it in a ``convention'' register.

%% Similarly, we
Since we expect the reader to be a mathematician, not necessarily an
engineer, we'll also add ``lesson'' registers. These contain programming
practices (like the use of unit tests on simple examples) which
mathematicians may be unaware of, but programmers should know.

\section*{Exercises and Examples}

Exercises are littered throughout the text. The reader is encouraged to
do them! If I have borrowed an exercise from a text, I try to cite it
with sufficient information. To alert the reader that something is an
exercise, a triangle is inserted into the left-margin.

Examples are worked exercises, or they illustrate a point. For example,
in the first chapter, examples are used to demonstrate floating-point
arithmetic oddities.

After the first chapter (on floating-point arithmetic), examples may be
recurring, in the sense that: we evaluate several ``competing''
algorithms on several examples, but we introduce each algorithm one at a
time.

\section*{Outline of Topics}

The three parts to this text correspond to the sequence which I learned
numerical analysis.

The first part introduces the basics of numerical analysis. Chapter one begins
with a thorough discussion of floating-point arithmetic, starting with
an ``idealized floating-point'' where the number system intuitively
is scientific notation $m\times 10^{e}$ where $1\leq m<10$ is a
$t$-digit number, and $e\in\ZZ$ is an arbitrary integer (this is the
``idealized'' part). Then we transition to real \ieee-754
floating-point, with emphasis on results in base $\beta=2$. We treat it
axiomatically as a number system with certain mathematical
operations. But we also provide explicit algorithms describing how the
binary arithmetic operators work.

Chapter two discusses the first toy problem: evaluating a polynomial at
a point. This toy problem demonstrates na\"{\i}ve algorithms actually
lose precision.

Chapter three discusses polynomial interpolation. Given some function
$f(x)$ and some domain of interest, say $[a,b]$, can we find $n+1$ points
$x_{0}$, \dots, $x_{n}$ such that we can construct a polynomial $P(x)$
where $P(x_{j})=f(x_{j})$ and $\deg(P)\leq n$? Can we do it to minimize
the ``error'' of such an approximation? We use the results from the
second chapter to evaluate it efficiently.

Chapter four discusses numerical differentiation. This is another
``playground'' where we can compare the numerical schemes to exact
answers, to measure the error. It also serves as a basis for reasoning
in part three, when we get to numerical differential equations.

Chapter five discusses quadrature (a.k.a., numerical integration).
This is where we really take a leap of faith, because we can't always
solve an integral in closed-form.

\bigbreak
Part two discusses numerical linear algebra. This is a beautiful
subject, which can be used as a bridge between finite-dimensional linear
algebra and functional analysis. There are too many wonderful texts on
this subject already.

\bigbreak
Part three discusses numerical differential equations.

\subsection*{Starred Sections}

Optional or advanced sections which can be skipped I've indicated with
an asterisk or star, e.g.,
\S\ref{subsec:computer:extended-real-number-system} ``*Extended Real
Number System''. This is a topic I feel is useful for the reader, but
the reader may skip it without shame or guilt.

\section*{License}

The code produced here is licensed under the Clear BSD License:

\bigskip

Copyright \copyright\ 2021 Alex Nelson

All rights reserved.

Redistribution and use in source and binary forms, with or without
modification, are permitted (subject to the limitations in the
disclaimer below) provided that the following conditions are met:

\begin{itemize}
\item Redistributions of source code must retain the above copyright
     notice, this list of conditions and the following disclaimer.
\item Redistributions in binary form must reproduce the above copyright
     notice, this list of conditions and the following disclaimer in the
     documentation and/or other materials provided with the
     distribution.
\item Neither the name of the copyright holder nor the names of its
     contributors may be used to endorse or promote products derived
     from this software without specific prior written permission.
\end{itemize}

NO EXPRESS OR IMPLIED LICENSES TO ANY PARTY'S PATENT RIGHTS ARE GRANTED
BY THIS LICENSE. THIS SOFTWARE IS PROVIDED BY THE COPYRIGHT HOLDERS AND
CONTRIBUTORS ``AS IS'' AND ANY EXPRESS OR IMPLIED WARRANTIES, INCLUDING,
BUT NOT LIMITED TO, THE IMPLIED WARRANTIES OF MERCHANTABILITY AND
FITNESS FOR A PARTICULAR PURPOSE ARE DISCLAIMED. IN NO EVENT SHALL THE
COPYRIGHT HOLDER OR CONTRIBUTORS BE LIABLE FOR ANY DIRECT, INDIRECT,
INCIDENTAL, SPECIAL, EXEMPLARY, OR CONSEQUENTIAL DAMAGES (INCLUDING, BUT
NOT LIMITED TO, PROCUREMENT OF SUBSTITUTE GOODS OR SERVICES; LOSS OF
USE, DATA, OR PROFITS; OR BUSINESS INTERRUPTION) HOWEVER CAUSED AND ON
ANY THEORY OF LIABILITY, WHETHER IN CONTRACT, STRICT LIABILITY, OR TORT
(INCLUDING NEGLIGENCE OR OTHERWISE) ARISING IN ANY WAY OUT OF THE USE OF
THIS SOFTWARE, EVEN IF ADVISED OF THE POSSIBILITY OF SUCH DAMAGE.

\endinput
